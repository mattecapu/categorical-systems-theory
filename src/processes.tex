\section{Composition}
The starting point for defining a theory of systems is defining a double category of composition patterns such systems use to interact with one another.
In practice, defining the systems and defining the composition patterns are activities that influence one another.
Systems are made out of composition patterns themselves, and it's usually them we have in mind when we approach a formalization problem.
So one often starts by asking how the systems at hand could possibly be composed together.

Composition patterns usually form an operad.\footnote{By `operad' we mean `coloured operad' which means `multicategory'.}
Operads are indeed ways to specify how `small things fit into larger things', i.e.~they are \emph{theories of composition}.
Most importantly, they give meaning to various kinds of wiring diagrams~\cite{spivak2013operad,vagner2014algebras, libkind2021operadic}.

One can also see composition patterns as \emph{processes} that extend a given system with further dynamics, possibly gathering many systems in one.
This point of view can be more natural from a `European' point of view, more acquainted with string diagrams rather than wiring diagrams.
In fact one can see double categories of composition patterns as higher-dimensional extensions of the process theories of Abramsky, Coecke, Gogioso~\cite{abramsky2004categorical, coecke2018picturing}.

\begin{definition}
	A \textbf{composition theory} is a symmetric monoidal double category with attitude.
\end{definition}

Hence any monoidal double category can be a composition theory if we have a convincing interpretation of it as such.
At this level of generality there's no justification for limiting this definition further.
Of course any specific doctrine of systems can make more opinionated choices on which monoidal double categories to admit, and we will see plenty of examples of this.

\begin{remark}
	Rather than defining double categories of processes to be monoidal, one should arguably define them to be multicategories.
	Thus a nicer, if more exotic, definition of composition theory would be that of a `multicategory internal to categories'.
	This is the reason here, following the custom recently adopted by Myers, we often denote processes as having many inputs.
	With our definitions, one might interpret the notation $I_1, \ldots, I_n$ as denoting a monoidal product, i.e.~we denote with `$, $' the monoidal product on processes.
\end{remark}

A composition theory $\Processes$ thus unpacks as follows:
\begin{enumerate}
	\item There are a number of objects $I, J, K, \ldots$ which are \textbf{interfaces} of the processes.%, or the \textbf{boxes} of the composition patterns.
	\item There are horizontal maps $h: I \horto J$ which are \textbf{maps of interfaces}, without dynamic content, they simply compare different interfaces with each other.\footnote{One could call these `algebraic maps of interfaces', as their role is to provide morphisms that \emph{preserve} the interface structure, in order to compare them.}
	\begin{equation}
		% https://q.uiver.app/?q=WzAsMixbMCwwLCJJIl0sWzEsMCwiSiJdLFswLDEsImgiXV0=
		\begin{tikzcd}[ampersand replacement=\&]
			I \& J
			\arrow["h", from=1-1, to=1-2]
		\end{tikzcd}
	\end{equation}
	\item There are vertical maps $p : I_1, \ldots, I_n \verto K$  which are the \textbf{processes}, connecting interfaces in some way.\footnote{Again, one might call these `geometric maps of interfaces', as they \emph{reflect} structure in many example we care about}
	\begin{equation}
		% https://q.uiver.app/?q=WzAsMixbMCwwLCJJIl0sWzAsMSwiSyJdLFswLDEsInAiLDIseyJzdHlsZSI6eyJib2R5Ijp7Im5hbWUiOiJiYXJyZWQifX19XV0=
		\begin{tikzcd}[ampersand replacement=\&]
			I_1, \ldots, I_n \\
			K
			\arrow["p"', "\bullet"{marking}, from=1-1, to=2-1]
		\end{tikzcd}
	\end{equation}
	\item There are squares $\alpha : p \twoto q$ which are \textbf{maps of processes} supported by given maps of interfaces:
	\begin{equation}
		% https://q.uiver.app/?q=WzAsNCxbMCwwLCJJIl0sWzAsMSwiSyJdLFsxLDAsIkoiXSxbMSwxLCJMIl0sWzAsMiwiaCJdLFswLDEsInAiLDIseyJzdHlsZSI6eyJib2R5Ijp7Im5hbWUiOiJiYXJyZWQifX19XSxbMSwzLCJrIiwyXSxbMiwzLCJxIiwwLHsic3R5bGUiOnsiYm9keSI6eyJuYW1lIjoiYmFycmVkIn19fV0sWzUsNywiXFxhbHBoYSIsMCx7InNob3J0ZW4iOnsic291cmNlIjoyMCwidGFyZ2V0IjoyMH19XV0=
		\begin{tikzcd}[ampersand replacement=\&]
			{I_1, \ldots, I_n} \& {J_1, \ldots, J_n} \\
			K \& L
			\arrow["{h_1, \ldots,\, h_n}", from=1-1, to=1-2]
			\arrow[""{name=0, anchor=center, inner sep=0}, "p"', "\bullet"{marking}, from=1-1, to=2-1]
			\arrow["k"', from=2-1, to=2-2]
			\arrow[""{name=1, anchor=center, inner sep=0}, "q", "\bullet"{marking}, from=1-2, to=2-2]
			\arrow["\alpha", shorten <=15pt, shorten >=15pt, Rightarrow, from=0, to=1]
		\end{tikzcd}
	\end{equation}
\end{enumerate}

A very important takeway behind this definition is that it distinguishes morphisms \emph{as} processes from morphisms \emph{of} processes.
In fact, the usual yoga of process theories (which amount to symmetric monoidal `single' categories) only focuses on the compositional properties of processes (they compose like morphisms).
But in category theory we know that if we want to study something, we need to study morphisms between them!

\begin{example}[Alphabets]
\label{ex:alphabets}
	Let $\Alph = \FinSet^\uparrow$ be the double category of alphabets and alphabet reductions, and maps thereof.
	Its objects are finite sets of symbols we call `alphabets'.
	Its horizontal maps are maps of finite sets.
	Its vertical maps are \emph{alphabet reductions}, which are map of finite sets in the opposite direction:
	\begin{equation}
		p : \Sigma \verto \Sigma' \in \FinSet(\Sigma', \Sigma)
	\end{equation}
	Squares are commutative squares:
	\begin{equation}
		% file:///home/jsb20179/data/software/quiver/src/index.html?q=WzAsNCxbMCwwLCJcXFNpZ21hIl0sWzAsMSwiXFxTaWdtYSciXSxbMSwwLCJcXFhpIl0sWzEsMSwiXFxYaSciXSxbMSwwLCJwIl0sWzMsMiwicSIsMl0sWzAsMiwiaCJdLFsxLDMsImsiLDJdXQ==
		\begin{tikzcd}[ampersand replacement=\&]
			\Sigma \& \Xi \\
			{\Sigma'} \& {\Xi'}
			\arrow["p", from=2-1, to=1-1]
			\arrow["q"', from=2-2, to=1-2]
			\arrow["h", from=1-1, to=1-2]
			\arrow["k"', from=2-1, to=2-2]
		\end{tikzcd}
	\end{equation}
\end{example}

\begin{example}[Bidirectional processes]
\label{ex:lenses}
	Consider the following double category of \emph{deterministic bidirectional processes} $\dblLens$:
	\begin{enumerate}
		\item interfaces are given by pairs or sets $\lens{A^-}{A^+}$, whose monoidal product is given by componentwise product,
		\item processes $\lens{A^-}{A^+} \opticto \lens{C^-}{C^+}$ are given by \textbf{lenses} $\lens{f^\sharp}{f}$, comprised of a \emph{get part} $f : A^+ \to C^+$ and a \emph{put part} $f^\sharp : A^+ \times C^- \to A^-$,
		\item maps of interfaces $\lens{A^-}{A^+} \chartto \lens{B^-}{B^+}$ are given by \textbf{charts} $\lens{g^\flat}{g}$, comprised of a \emph{states part} $g : A^+ \to B^+$ and a \emph{directions part} $g^\flat : A^+ \times A^- \to B^-$,
		\item behaviours are given by arrangements
		\begin{equation}
		\label{eq:det-behav}
			% https://q.uiver.app/?q=WzAsNCxbMCwwLCJcXGxlbnN7QV4tfXtBXit9Il0sWzAsMSwiXFxsZW5ze0NeLX17Q14rfSJdLFsxLDEsIlxcbGVuc3tEXi19e0ReK30iXSxbMSwwLCJcXGxlbnN7Ql4tfXtCXit9Il0sWzAsMywiZyIsMix7Im9mZnNldCI6MX1dLFswLDEsImYiLDIseyJvZmZzZXQiOjF9XSxbMywyLCJrIiwyLHsib2Zmc2V0IjoxfV0sWzEsMiwiaCIsMix7Im9mZnNldCI6MX1dLFsxLDAsImZeXFxzaGFycCIsMix7Im9mZnNldCI6MX1dLFsyLDMsImteXFxzaGFycCIsMix7Im9mZnNldCI6MX1dLFswLDMsImdeXFxmbGF0IiwwLHsib2Zmc2V0IjotMX1dLFsxLDIsImheXFxmbGF0IiwwLHsib2Zmc2V0IjotMX1dXQ==
			\begin{tikzcd}[ampersand replacement=\&]
				{\lens{A^-}{A^+}} \& {\lens{B^-}{B^+}} \\
				{\lens{C^-}{C^+}} \& {\lens{D^-}{D^+}}
				\arrow["g"', shift right=1, from=1-1, to=1-2]
				\arrow["f"', shift right=1, from=1-1, to=2-1]
				\arrow["k"', shift right=1, from=1-2, to=2-2]
				\arrow["h"', shift right=1, from=2-1, to=2-2]
				\arrow["{f^\sharp}"', shift right=1, from=2-1, to=1-1]
				\arrow["{k^\sharp}"', shift right=1, from=2-2, to=1-2]
				\arrow["{g^\flat}", shift left=1, from=1-1, to=1-2]
				\arrow["{h^\flat}", shift left=1, from=2-1, to=2-2]
			\end{tikzcd}
		\end{equation}
		such that for every $a^+ \in A^+$ and $c^- \in C^-$:
		\begin{eqalign}
			k(g(a^+)) &= h(f(a^+)),\\
			g^\flat(a^+, f^\sharp(a^+, c^-)) &= k^\sharp(g(a^+), h^\flat(f(a^+), c^-)).
		\end{eqalign}
	\end{enumerate}
	The last conditions are hard to parse formally, but basically they say that both squares one can spot in~\eqref{eq:det-behav} commute.
	Concretely, we have two bidirectional processes $\lens{f^\sharp}{f}$ and $\lens{k^\sharp}{k}$ and a way to map between their interfaces.
	We are then asking that their dynamics commute with such maps.
\end{example}

\begin{remark}
	Viewed as composition patterns, lenses are algebras of the operad of wiring diagrams~\cite{spivak2013operad}.
	Thus they represent ways to wire a number of boxes into a larger box:
	\begin{figure}[H]
		\matteo{wiring diagram}
	\end{figure}
\end{remark}

\begin{example}
\label{ex:f-lenses}
	The previous example can be generalized greatly by employing $F$-lenses~\cite{spivak_generalized_2019}.
	These are lenses in which the backward part is dependent on the forward part in a way specified by an indexed category $F$.
	Intuitively, this correspond to a wiring pattern which can change dependening on the what flows in the wires (see~\cite{spivak_poly_2020}).

	The definitions of $F$-lenses and $F$-charts are substantially identical to the ones above, as well as that for the squares (except the `commutativity condition' is now harder to eyeball).
	We gather some examples here:
	\matteo{add dependent variants for other lenses}
	\begin{table}
		\begin{tabular}{l|l|l}
			\textbf{lenses} & \textbf{category} & $F$\\
			\hline
			\textbf{deterministic} & $\cat C$ cartesian monoidal & $\Set/_{\sf proj} -$ (or $\coKl(- \times =)$) \\
			\textbf{deterministic (dependent)} & $\cat C$ finite limits & $\Set/ -$ \\
			\textbf{possibilistic} & $\cat E$ topos & $\biKl(- \times =, \pow)$ ($\pow$ powerset monad)\\
			\textbf{probabilistic} & $\Msbl$ & $\biKl(- \times =, \Delta)$ ($\Delta$ probability monad)\\
			\textbf{effectful} & $\cat C$ cartesian monoidal & $\biKl(- \times =, M)$ ($M$ commutative monad)\\
			\textbf{differential (Euclidean)} & $\Euc$ & $\Euc/_{\sf subm}-$\\
			\textbf{differential (general)} & $\Smooth$ & $\Smooth/_{\sf subm}-$
		\end{tabular}
		\caption{Flavours of $F$-lenses. We put them in pairs, with the first being `simple' lenses, like those in~\cref{ex:lenses}, and the second `dependent'.}
	\end{table}
\end{example}

\begin{example}[Behavioural theory]
\label{ex:behav-processes}
	Given any category with finite limits\footnote{Or, more directly for our purposes, one whose codomain functor $\cod:\cat C^\downarrow \to \cat C$ is a fibration.} $\cat C$, there is a double category $\Span(\cat C)$.
	Here, the objects are the same of $\cat C$, as well as the horizontal maps.
	The vertical maps are given by spans instead, and the squares by morphisms of spans:v
	\begin{equation}
		% https://q.uiver.app/?q=WzAsNixbMCwwLCJJIl0sWzEsMCwiSiJdLFswLDEsIlgiXSxbMSwxLCJZIl0sWzAsMiwiSSciXSxbMSwyLCJKJyJdLFsyLDQsInBfciIsMl0sWzMsNSwicV9yIl0sWzIsMCwicF9cXGVsbCJdLFszLDEsInFfXFxlbGwiLDJdLFsyLDMsIlxcYWxwaGEiLDFdLFs0LDUsImsiLDJdLFswLDEsImgiXV0=
		\begin{tikzcd}[ampersand replacement=\&]
			I \& J \\
			X \& Y \\
			{I'} \& {J'}
			\arrow["{p_r}"', from=2-1, to=3-1]
			\arrow["{q_r}", from=2-2, to=3-2]
			\arrow["{p_\ell}", from=2-1, to=1-1]
			\arrow["{q_\ell}"', from=2-2, to=1-2]
			\arrow["\alpha"{description}, from=2-1, to=2-2]
			\arrow["k"', from=3-1, to=3-2]
			\arrow["h", from=1-1, to=1-2]
		\end{tikzcd}
	\end{equation}
	We call these double categories `behavioural theories of processes' since they model processes only by their input-output relation, hence by the behaviour they expose.
	Of course this is only an attitude.
\end{example}

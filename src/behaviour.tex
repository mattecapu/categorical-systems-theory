\section{Behaviours}
\label{sec:behaviour}
Behaviours in CST are simply maps into a behavioural theory. By default, we consider behaviours valued in $\BSys(\Set)$, but one can consider `structured behaviours' into $\BSys(\cat C)$ when one wants to keep track of extra structure on the set of behaviours of a system.

\begin{definition}
	A \textbf{theory of behaviour} or just \textbf{behaviour} for a systems theory $\Sys$ valued in the Cartesian category $\cat C$ is a map of system theories $\systh B:\Sys \to \BSys(\cat C)$:
	\begin{equation}
		% https://q.uiver.app/?q=WzAsNSxbMCwwLCJcXFByb2Nlc3Nlc15cXHRvcCJdLFswLDIsIlxcU3BhbihcXGNhdCBDKV5cXHRvcCJdLFsyLDEsIlxcZGJsQ2F0Il0sWzEsMF0sWzEsMl0sWzAsMSwiQl5cXHRvcCIsMl0sWzAsMiwiXFxTeXMiLDAseyJjdXJ2ZSI6LTF9XSxbMSwyLCJcXGNhdCBDLy0iLDIseyJjdXJ2ZSI6MX1dLFszLDQsIkJeXFxmbGF0IiwwLHsib2Zmc2V0Ijo1LCJzaG9ydGVuIjp7InNvdXJjZSI6MzAsInRhcmdldCI6MzB9LCJsZXZlbCI6Mn1dXQ==
		\begin{tikzcd}[ampersand replacement=\&, sep=small]
			{\Processes^\top} \& {} \\
			\&\& \dblCat \\
			{\Span(\cat C)^\top} \& {}
			\arrow["{B^\top}"', from=1-1, to=3-1]
			\arrow["\Sys", curve={height=-6pt}, from=1-1, to=2-3]
			\arrow["{\cat C/-}"', curve={height=6pt}, from=3-1, to=2-3]
			\arrow["{B^\flat}"', shift right=4, shorten <=10pt, shorten >=15pt, Rightarrow, from=1-2, to=3-2]
		\end{tikzcd}
	\end{equation}
	Hence $\systh B$ is given by a lax double functor
	\begin{equation}
		B : \Processes \longto \Span(\cat C)
	\end{equation}
	and by a vertical lax-natural transformation whose components are given, for every $I: \Processes$, by
	\begin{equation}
		B^\flat_I : \Sys(I) \longto \cat C/I.
	\end{equation}
\end{definition}

The intuition behind this definition is the following. First of all, $B$ maps interfaces to objects of observations we can make about them.
Every theory of behaviour gives its own notion of what `observation' means: it could be mere points, it could be distributions, it could be sequences of observations, and so on (examples will prove the variety this definition allows).
Then, we map every system to an object of states (which are \emph{observations we can make about the ways the system can be}) and a map that displays the dependency of said state on the observations on its interface (thus \emph{observations we can make about the ways the system can change}). This mapping is given by

\begin{example}[States]
\label{ex:states}
	One of the simplest notion of behaviour for a system is given by the idea of `state space'. This is the very classical stance that what can be observed about a system is its permanence in a state which varies in a prescribed space.
	So suppose $\cat C$ is some category of spaces\footnote{Following Lawvere \cite{lawvere_categories_1992}, we consider `cartesian \& extensive' to be satisfying notion of \emph{category of spaces}.} and let's consider the theory of coalgebras on $\cat C$: $\Coalg_{\cat C} : \dblEnd^\top \unilaxto \dblCat$.

	Then there is a theory of behaviour:
	\begin{equation}
		% https://q.uiver.app/?q=WzAsNSxbMCwwLCJcXGRibGNhdHtFbmR9XlxcdG9wIl0sWzAsMiwiXFxTcGFuKFxcY2F0IFMpXlxcdG9wIl0sWzIsMSwiXFxkYmxDYXQiXSxbMSwwXSxbMSwyXSxbMCwxLCIxXlxcdG9wIiwyXSxbMCwyLCJcXENvYWxnIiwwLHsiY3VydmUiOi0xfV0sWzEsMiwiXFxjYXQgUy8tIiwyLHsiY3VydmUiOjF9XSxbMyw0LCJcXHN0YXRlcyIsMix7ImxhYmVsX3Bvc2l0aW9uIjo0MCwib2Zmc2V0Ijo0LCJzaG9ydGVuIjp7InNvdXJjZSI6MzAsInRhcmdldCI6NTB9LCJsZXZlbCI6Mn1dXQ==
		\begin{tikzcd}[ampersand replacement=\&,sep=small]
			{\dblEnd^\top} \& {} \\
			\&\& \dblCat \\
			{\Span(\cat C)^\top} \& {}
			\arrow["{1^\top}"', from=1-1, to=3-1]
			\arrow["\Coalg_{\cat C}", curve={height=-6pt}, from=1-1, to=2-3]
			\arrow["{\cat C/-}"', curve={height=6pt}, from=3-1, to=2-3]
			\arrow["\states"'{pos=0.4}, shift right=4, shorten <=10pt, shorten >=15pt, Rightarrow, from=1-2, to=3-2]
		\end{tikzcd}
	\end{equation}
	which is trivial on the interfaces and maps coalgebras to their carriers:
	\begin{eqalign}
		\states_F : \Coalg_{\cat C}(F) &\longto \cat C\\
					(S, S \nto{\delta} FS) &\longmapsto S.
	\end{eqalign}

	Clearly such a theory of behaviour can also be defined for Moore machines. Let $\Moore_{(F, T)} : \dblLens^\top \unilaxto \dblCat$ be a theory of Moore machines, then there is again a theory of behaviour
	\begin{equation}
		% https://q.uiver.app/?q=WzAsNSxbMCwwLCJcXGRibExlbnNeXFx0b3AiXSxbMCwyLCJcXFNwYW4oXFxjYXQgQyleXFx0b3AiXSxbMiwxLCJcXGRibENhdCJdLFsxLDBdLFsxLDJdLFswLDEsIjFeXFx0b3AiLDJdLFswLDIsIlxcTW9vcmVfeyhGLCBUKX0iLDAseyJjdXJ2ZSI6LTF9XSxbMSwyLCJcXGNhdCBDLy0iLDIseyJjdXJ2ZSI6MX1dLFszLDQsIlxcc3RhdGVzIiwyLHsibGFiZWxfcG9zaXRpb24iOjQwLCJvZmZzZXQiOjQsInNob3J0ZW4iOnsic291cmNlIjozMCwidGFyZ2V0Ijo1MH0sImxldmVsIjoyfV1d
		\begin{tikzcd}[ampersand replacement=\&,sep=small]
			{\dblLens^\top} \& {} \\
			\&\& \dblCat \\
			{\Span(\cat C)^\top} \& {}
			\arrow["{1^\top}"', from=1-1, to=3-1]
			\arrow["{\Moore_{(F, T)}}", curve={height=-6pt}, from=1-1, to=2-3]
			\arrow["{\cat C/-}"', curve={height=6pt}, from=3-1, to=2-3]
			\arrow["\states"'{pos=0.4}, shift right=4, shorten <=10pt, shorten >=16pt, Rightarrow, from=1-2, to=3-2]
		\end{tikzcd}
	\end{equation}
	which is trivial on the interfaces and maps Moore machines to their state spaces:
	\begin{eqalign}
		\states_{\lens{I}{O}} : \Moore_{(F,T)}\lens{I}{O} &\longto \cat C\\
					(S, \lens{\update}{\expose} : \lens{TS}{S} \opticto \lens{I}{O}) &\longmapsto S.
	\end{eqalign}
\end{example}

\begin{example}[Reachability]
\label{ex:reachability}
	Labelled transition systems admit a notion of behaviour given by reachability:
	\begin{equation}
		\begin{tikzcd}[ampersand replacement=\&,sep=small]
			{\Alph^\top} \& {} \\
			\&\& \dblCat \\
			{\Span(\Set)^\top} \& {}
			\arrow["{{(-)^*}^\top}"', from=1-1, to=3-1]
			\arrow["{\Trans}", curve={height=-6pt}, from=1-1, to=2-3]
			\arrow["{\Set/-}"', curve={height=6pt}, from=3-1, to=2-3]
			\arrow["R"'{pos=0.4}, shift right=4, shorten <=10pt, shorten >=16pt, Rightarrow, from=1-2, to=3-2]
		\end{tikzcd}
	\end{equation}
	where
	\begin{enumerate}
		\item $\Trans : \Alph^\top \unilaxto \dblCat$ is the theory of systems introduced in~\cref{ex:trans-sys};
		\item $(-)^* : \Alph \to \Span(\Set)$ sends a finite alphabet $\Sigma$ to the free monoid $\Sigma^*$, an alphabet mapping $h:\Sigma \to \Sigma'$ to the morphism of monoids $h^* : \Sigma^* \to {\Sigma'}^*$, an alphabet reduction $p: \Sigma \from \Xi$ to the span $\Sigma^* \nfrom{p^*} \Xi^* \equalto \Xi^*$, and a commutative square to an obvious morphism of spans;
		\item The component at $\Sigma : \Alph$ of $R$ is
		\begin{eqalign}
			R_\Sigma : \Trans(\Sigma) &\longto \Set/\Sigma^*\\
			S \times \Sigma \nto{\delta} S &\longmapsto \{s \nreachto{w}_\delta s', w \in \Sigma^* \} \nlongto{\pi_{\Sigma^*}} \Sigma^*
		\end{eqalign}
		where $s  \nreachto{w}_\delta s'$ denotes a triple $(w, s, s') \in \Sigma^* \times S \times S$ such that $s'$ can be reached from $s$ by following the transitions $w_1 \cdots w_n = w$ determined by the transition map $\delta$.
		\item The vertical lax natural structure of $R$ is given, for each alphabet reduction $p: \Sigma \from \Xi$, by a natural transformation $R_p$ filling the square:
		\begin{equation}
			% https://q.uiver.app/?q=WzAsNCxbMCwwLCJcXFRyYW5zKFxcU2lnbWEpIl0sWzAsMSwiXFxUcmFucyhcXFhpKSJdLFsxLDAsIlxcU2V0L1xcU2lnbWEiXSxbMSwxLCJcXFNldC9cXFhpIl0sWzIsMywiXFxTZXQvcF4qIl0sWzAsMiwiUl9cXFNpZ21hIl0sWzEsMywiUl9cXFhpIiwyXSxbMCwxLCJcXFRyYW5zKHApIiwyXSxbNiw0LCJSX3AiLDAseyJvZmZzZXQiOi0zLCJzaG9ydGVuIjp7InNvdXJjZSI6MjAsInRhcmdldCI6MjB9fV1d
			\begin{tikzcd}[ampersand replacement=\&]
				{\Trans(\Sigma)} \& {\Set/\Sigma} \\
				{\Trans(\Xi)} \& {\Set/\Xi}
				\arrow[""{name=0, anchor=center, inner sep=0}, "{\Set/p^*}", from=1-2, to=2-2]
				\arrow["{R_\Sigma}", from=1-1, to=1-2]
				\arrow[""{name=1, anchor=center, inner sep=0}, "{R_\Xi}"', from=2-1, to=2-2]
				\arrow["{\Trans(p)}"', from=1-1, to=2-1]
				\arrow["{R_p}", shift left=3, shorten <=4pt, shorten >=4pt, Rightarrow, from=1, to=0]
			\end{tikzcd}
		\end{equation}
		We define it as follows. Given $S \times \Sigma \nto{\delta} S : \Trans(\Sigma)$, one gets, following the bottom path:
		\begin{equation}
			S \times \Sigma \nto{\delta} S
			\quad\mapsto\quad
			S \times \Xi \nto{S \times p} S \times \Sigma \nto{\delta} S
			\quad\mapsto\quad
			\begin{tikzcd}[ampersand replacement=\&]
				{\{s \nreachto{v}_{\Trans(p)(\delta)} s', v \in \Xi^*\}} \\
				{\Xi^*}
				\arrow["{\pi_{\Xi^*}}", from=1-1, to=2-1]
			\end{tikzcd}
		\end{equation}
		Following the top path, one gets instead:
		\begin{equation}
			S \times \Sigma \nto{\delta} S
			\quad\mapsto\quad
			\begin{tikzcd}[ampersand replacement=\&]
				{\{s \nreachto{w}_{\delta} s', w \in \Sigma^*\}} \\
				{\Sigma^*}
				\arrow["{\pi_{\Sigma^*}}", from=1-1, to=2-1]
			\end{tikzcd}
			\quad\mapsto\quad
			\begin{tikzcd}[ampersand replacement=\&, column sep=small]
				{\{s \nreachto{w}_{\delta} s', w = p^*(v), v \in \Xi^*\}} \& {\{s \nreachto{w}_{\delta} s', w \in \Sigma^*\}} \\
				{\Xi^*} \& {\Sigma^*}
				\arrow["{\pi_{\Sigma^*}}", from=1-2, to=2-2]
				\arrow["{p^*}"', from=2-1, to=2-2]
				\arrow["{\Set/p*(\pi_{\Sigma^*})}"', from=1-1, to=2-1]
				\arrow[from=1-1, to=1-2]
				\arrow["\lrcorner"{anchor=center, pos=0.125, rotate=45}, draw=none, from=1-1, to=2-2]
			\end{tikzcd}
		\end{equation}
		Thus the components of $R_p$ are:
		\begin{equation}
			% https://q.uiver.app/?q=WzAsMyxbMSwxLCJcXFhpXioiXSxbMiwwLCJcXHtzIFxcbnJlYWNodG97d31fe1xcZGVsdGF9IHMnLCB3ID0gcF4qKHYpLCB2IFxcaW4gXFxYaV4qXFx9Il0sWzAsMCwiXFx7cyBcXG5yZWFjaHRve3Z9X3tcXFRyYW5zKHApKFxcZGVsdGEpfSBzJywgdiBcXGluIFxcWGleKlxcfSJdLFsxLDBdLFsyLDBdLFsyLDEsIlJfcChcXGRlbHRhKSJdXQ==
			\begin{tikzcd}[ampersand replacement=\&, sep=scriptsize]
				{\{s \nreachto{v}_{\Trans(p)(\delta)} s', v \in \Xi^*\}} \&\& {\{s \nreachto{w}_{\delta} s', w = p^*(v), v \in \Xi^*\}} \\
				\& {\Xi^*}
				\arrow[from=1-3, to=2-2]
				\arrow[from=1-1, to=2-2]
				\arrow["{R_p(\delta)}", from=1-1, to=1-3]
			\end{tikzcd}
		\end{equation}
		simply by setting $R_p(\delta)(s \nreachto{v}_{\Trans(p)(\delta)} s') = s \nreachto{p(v)}_{\delta} s'$. Clearly this assignment is natural in $\delta$ (since maps of labelled transition systems commute with reachability) and $R_p \comp R_q = R_{p \comp q}$ (vertical functoriality).
		\item The naturality squares for $R$ are given, for each horizontal $h:\Sigma \to \Sigma'$ in $\Alph$, by
		\begin{equation}
			% https://q.uiver.app/?q=WzAsNCxbMCwwLCJcXFRyYW5zKFxcU2lnbWEpIl0sWzAsMSwiXFxUcmFucyhcXFNpZ21hJykiXSxbMSwwLCJcXFNldC9cXFNpZ21hXioiXSxbMSwxLCJcXFNldC97XFxTaWdtYSd9XioiXSxbMCwxLCJcXFRyYW5zKGgpIiwyLHsic3R5bGUiOnsiYm9keSI6eyJuYW1lIjoiYmFycmVkIn19fV0sWzIsMywiXFxTZXQvaCIsMCx7InN0eWxlIjp7ImJvZHkiOnsibmFtZSI6ImJhcnJlZCJ9fX1dLFswLDIsIlJfXFxTaWdtYSJdLFsxLDMsIlJfe1xcU2lnbWEnfSIsMl0sWzQsNSwiUl9oIiwxLHsic2hvcnRlbiI6eyJzb3VyY2UiOjIwLCJ0YXJnZXQiOjIwfX1dXQ==
			\begin{tikzcd}[ampersand replacement=\&]
				{\Trans(\Sigma)} \& {\Set/\Sigma^*} \\
				{\Trans(\Sigma')} \& {\Set/{\Sigma'}^*}
				\arrow[""{name=0, anchor=center, inner sep=0}, "{\Trans(h)}"', "\shortmid"{marking}, from=1-1, to=2-1]
				\arrow[""{name=1, anchor=center, inner sep=0}, "{\Set/h^*}", "\shortmid"{marking}, from=1-2, to=2-2]
				\arrow["{R_\Sigma}", from=1-1, to=1-2]
				\arrow["{R_{\Sigma'}}"', from=2-1, to=2-2]
				\arrow["{R_h}"{description}, shorten <=10pt, shorten >=10pt, Rightarrow, from=0, to=1]
			\end{tikzcd}
		\end{equation}
		defined as
		\begin{eqalign}
			R_\alpha(\delta, \zeta) : \Trans(h)(\delta, \zeta) &\longto \Set/k^*(R_\Sigma(\delta), R_{\Sigma^*}(\zeta))\\
			% https://q.uiver.app/?q=WzAsNCxbMCwwLCJTIFxcdGltZXMgXFxTaWdtYSAiXSxbMCwxLCJTIl0sWzEsMCwiVCBcXHRpbWVzIFxcU2lnbWEnIl0sWzEsMSwiVCJdLFsxLDMsIlxcdmFycGhpIl0sWzAsMiwiXFx2YXJwaGkgXFx0aW1lcyBoIl0sWzAsMSwiXFxkZWx0YSIsMl0sWzIsMywiXFx6ZXRhIl1d
			\begin{tikzcd}[ampersand replacement=\&]
				{S \times \Sigma } \& {T \times \Sigma'} \\
				S \& T
				\arrow["\varphi", from=2-1, to=2-2]
				\arrow["{\varphi \times h}", from=1-1, to=1-2]
				\arrow["\delta"', from=1-1, to=2-1]
				\arrow["\zeta", from=1-2, to=2-2]
			\end{tikzcd}
			&\longmapsto
			\begin{tikzcd}[ampersand replacement=\&]
				{\{s \nreachto{v}_\delta s'\}} \& {\{s \nreachto{w}_\zeta s'\}} \\
				{\Sigma^*} \& {{\Sigma'}^*}
				\arrow["{h^*}"', from=2-1, to=2-2]
				\arrow["{\pi_{{\Sigma'}^*}}", from=1-2, to=2-2]
				\arrow["{\pi_{\Sigma^*}}"', from=1-1, to=2-1]
				\arrow["{R_h(\varphi)}", from=1-1, to=1-2]
			\end{tikzcd}
		\end{eqalign}
		where $R_h(s \nreachto{v}_\delta s') = \varphi(s) \nreachto{h^*(v)}_\zeta \varphi(s')$. This is well-defined since, by definition, $\varphi$ preserve transitions. We leave the reader to prove horizontal functoriality.
	\end{enumerate}
\end{example}

\begin{example}[Fixpoints]
	\matteo{TODO}
\end{example}

\begin{example}[Trajectories]
	\matteo{TODO} % integral curves of vector fields, traces of open dynamical systems
\end{example}

% \begin{example}[Coalgebraic behaviour]
% 	Let $\cat C$ be a cartesian category. Let $\dblEnd^\nu(\cat C)$ denote the double category of endofunctors described in \cref{ex:coalgebras} but restricted to those endofunctors that admit a final coalgebra.
% 	We thus end up with a restricted theory of coalgebras $\Coalg^\nu : \theoryon{\dblEnd^\nu}$. On this, we can define the following theory of behaviour:
% 	\begin{equation}
% 		% https://q.uiver.app/?q=WzAsNSxbMCwwLCJ7XFxkYmxFbmReXFxudX1eXFx0b3AiXSxbMCwyLCJcXFNwYW4oXFxjYXQgQyleXFx0b3AiXSxbMiwxLCJcXGRibENhdCJdLFsxLDBdLFsxLDJdLFswLDEsIlxcbnUiLDJdLFswLDIsIlxcQ29hbGdeXFxudSIsMCx7ImN1cnZlIjotMX1dLFsxLDIsIlxcY2F0IEMvLSIsMix7ImN1cnZlIjoxfV0sWzMsNCwiISIsMix7ImxhYmVsX3Bvc2l0aW9uIjo0MCwib2Zmc2V0Ijo0LCJzaG9ydGVuIjp7InNvdXJjZSI6MzAsInRhcmdldCI6NTB9LCJsZXZlbCI6Mn1dXQ==
% 		\begin{tikzcd}[ampersand replacement=\&]
% 			{{\dblEnd^\nu(\cat C)}^\top} \& {} \\
% 			\&\& \dblCat \\
% 			{\Span(\cat C)^\top} \& {}
% 			\arrow["\nu"', from=1-1, to=3-1]
% 			\arrow["{\Coalg^\nu}", curve={height=-6pt}, from=1-1, to=2-3]
% 			\arrow["{\cat C/-}"', curve={height=6pt}, from=3-1, to=2-3]
% 			\arrow["{!}"'{pos=0.4}, shift right=4, shorten <=10pt, shorten >=15pt, Rightarrow, from=1-2, to=3-2]
% 		\end{tikzcd}
% 	\end{equation}
% 	This is given by
% 	\begin{enumerate}
% 		\item The lax double functor which sends an endofunctor $F : \dblEnd^\nu$ to the carrier of its final coalgebra $\nu F : \cat C$.

% 		Since $\nu$ is functorial on $\End^\nu(\cat C)$, we can map natural transformations $p:F \twoto G$ (the vertical arrows in $\dblEnd^\nu(\cat C)$) to `functional spans' $\nu F \equalto \nu F \nto{\nu p} \nu G$, and this is again functorial.
% 		In the horizontal direction, we have to prove a generalized chart $\lens{h^\flat}{h} : F \horto G$  also induces a map $\nu \lens{h^\flat}{h} : \nu F \to \nu G$. Remember that such a generalized chart is given by a map $F1 \to G1$ and and by a natural transformation $h^\flat : f^*G \twoto F$ such that $h^\flat_1$ is invertible (without loss of generality, the identity).

% 		Adamek's theorem characterizes the carrier of final coalgebras as limits of the diagrams obtained by iterated application of a functor to the terminal object:
% 		\begin{equation}
% 			\cdots F^2(1) \nto{F!} F1 \nto{!} 1
% 		\end{equation}
% 		Clearly, a map like $h$ induces a morphism between the diagrams corresponding to $F$ and $G$, which in turn induces a morphism $\nu F \to \nu G$.

% 		\item The vertical lax-natural transformation
% 	\end{enumerate}
% \end{example}

\begin{example}[Languages]
	One of the first notions of `behaviour of a system' one encounters in computer science is that of \emph{language of a finite states machines}. These, indeed, give a theory of behaviours for the theory of finite states machine we described in~\cref{ex:fsm}.

	Given an FSM $\sys S = (S, \Sigma, \delta, S^*)$, its language is the subset of $\Sigma^*$ given by those words which, when fed to $\sys S$, leave it in an accepting state (one in $S^*$). More precisely, we first have to pick an initial state $s_0 \in S$, then we can start using $\delta$, inputting the given word one character at a time, until we are done. Thus the language of an $\sys S$ is really a family of subsets of $\Sigma^*$, given by the recursively-defined map
	\begin{eqalign}
		L_{\sys S} : S &\longto P\Sigma^*\\
		s_0 &\longmapsto \{\sigma_1\cdots\sigma_n \suchthat \sigma_2\cdots\sigma_n \in L_{\sys S}(\delta(s_0, \sigma_1)) \}.
	\end{eqalign}
	Notice this family of subsets can also be defined as a `bundle of sets' (a map we use for its fibers)
	\begin{equation}
		\Lambda(\sys S) := \{ (s_0, w) \mid \in w \in L_{\sys S}(s_0) \} \monoto S \times \Sigma^* \nlongto{\pi_{\Sigma^*}} \Sigma^* \in \Set/\Sigma^*.
	\end{equation}
	Thus we see there is a theory of behaviour
	\begin{equation}
		% https://q.uiver.app/?q=WzAsNSxbMCwwLCJcXEFscGheXFx0b3AiXSxbMCwyLCJcXFNwYW4oXFxTZXQpXlxcdG9wIl0sWzIsMSwiXFxkYmxDYXQiXSxbMSwwXSxbMSwyXSxbMCwxLCJ7KC0pXip9XlxcdG9wIiwyXSxbMCwyLCJcXEZTTSIsMCx7ImN1cnZlIjotMX1dLFsxLDIsIlxcU2V0Ly0iLDIseyJjdXJ2ZSI6MX1dLFszLDQsIkwiLDAseyJvZmZzZXQiOjUsInNob3J0ZW4iOnsic291cmNlIjozMCwidGFyZ2V0IjozMH0sImxldmVsIjoyfV1d
		\begin{tikzcd}[ampersand replacement=\&, sep=small]
			{\Alph^\top} \& {} \\
			\&\& \dblCat \\
			{\Span(\Set)^\top} \& {}
			\arrow["{{(-)^*}^\top}"', from=1-1, to=3-1]
			\arrow["\FSM", curve={height=-6pt}, from=1-1, to=2-3]
			\arrow["{\Set/-}"', curve={height=6pt}, from=3-1, to=2-3]
			\arrow["\Lambda", shift right=5, shorten <=10pt, shorten >=15pt, Rightarrow, from=1-2, to=3-2]
		\end{tikzcd}
	\end{equation}
	where
	\begin{enumerate}
		\item $(-)^* : \Alph \to \Span(\Set)$ is the same as in~\cref{ex:reachability},
		\item the component at $\Sigma : \Alph$ of $\Lambda$ is given by
		\begin{eqalign}
			L_\Sigma : \FSM(\Sigma) &\longto \Set/\Sigma^*\\
			(S, \Sigma, \delta, S^*) &\longmapsto \Lambda(\sys S) : \{ (s_0, w) \mid \in w \in L_{\sys S}(s_0) \} \to \Sigma^*
		\end{eqalign}
	\end{enumerate}

	The functoriality properties of these assignments can be proven by making the following observation. For each $n \in \N$ (including $0$), there are finite states machines $\sys{Fin\, n} = (\Fin\, n+1, \Fin\, n, \_+1, \{n\})$. Explicitly, these have $n+1$ states and are defined over an alphabet with $n$ symbols. The transition map is
	\begin{eqalign}
		\_ + 1 : \Fin\,n+1\ \times\ \Fin\, n &\longto \Fin\,n+1\\
		(k, i) &\longmapsto k+1
	\end{eqalign}
	and the only accepting state is $n \in \Fin\,n+1$.

	A map from this machine to another like $\sys S$, mediated by a map of alphabets $\sigma : \Fin\,n \to \Sigma$, looks like this:
	\begin{equation}
		% https://q.uiver.app/?q=WzAsNixbMCwwLCJcXEZpblxcLCBuKzEgXFx0aW1lcyBcXEZpblxcLG4iXSxbMCwyLCJcXEZpblxcLG4rMSJdLFsyLDAsIlMgXFx0aW1lcyBcXFNpZ21hIl0sWzIsMiwiUyJdLFsyLDMsIlNeKiJdLFswLDMsIlxce25cXH0iXSxbMiwzLCJcXGRlbHRhIiwyXSxbMCwxLCJcXF8rMSIsMl0sWzAsMiwicyJdLFsxLDMsIlxcc2lnbWEiXSxbNSw0LCJcXHN1YnNldGVxIiwzLHsic3R5bGUiOnsiYm9keSI6eyJuYW1lIjoibm9uZSJ9LCJoZWFkIjp7Im5hbWUiOiJub25lIn19fV0sWzUsMSwiXFxzdWJzZXRlcSIsMyx7InN0eWxlIjp7ImJvZHkiOnsibmFtZSI6Im5vbmUifSwiaGVhZCI6eyJuYW1lIjoibm9uZSJ9fX1dLFs0LDMsIlxcc3Vic2V0ZXEiLDMseyJzdHlsZSI6eyJib2R5Ijp7Im5hbWUiOiJub25lIn0sImhlYWQiOnsibmFtZSI6Im5vbmUifX19XV0=
		\begin{tikzcd}[ampersand replacement=\&, sep=small]
			{\Fin\, n+1 \times \Fin\,n} \&\& {S \times \Sigma} \\
			\\
			{\Fin\,n+1} \&\& S \\
			{\{n\}} \&\& {S^*}
			\arrow["\delta"', from=1-3, to=3-3]
			\arrow["{\_+1}"', from=1-1, to=3-1]
			\arrow["s \times \sigma", from=1-1, to=1-3]
			\arrow["s", from=3-1, to=3-3]
			\arrow["\subseteq"{marking}, draw=none, from=4-1, to=4-3]
			\arrow["\subseteq"{marking}, draw=none, from=4-1, to=3-1]
			\arrow["\subseteq"{marking}, draw=none, from=4-3, to=3-3]
		\end{tikzcd}
	\end{equation}
	Concretely, this is a map $s: \Fin\,n+1 \to S$ selecting a sequence of $n+1$ states $s_0, \ldots, s_n$ such that
	\begin{equation}
		\forall 0 \leq k \leq n-1,\ s_{k+1} = \delta(s_k, \sigma_k) \quad\text{and}\quad s_n \in S^*.
	\end{equation}
	In other words, this data amounts to a word $\sigma_1 \cdots \sigma_n \in \Sigma^*$ and a state $s_0 \in S$ such that $\sys S$ accepts $\sigma_1 \cdots \sigma_n$ in $n$ steps when starting from $s_0$.

	Notice this also works for the empty word: a map from $\sys {Fin\, 0}$ is given by just a choice of state $s_0 \in S^*$, meaning $\sys S$ accepts the empty word when starting from $s_0$.

	Therefore we see $((-)^*, \Lambda)$ is a sum of corepresentable behaviours, namely $\sum_{n \geq 0} \FSM(\sys {Fin\, n}, -)$. Observe the symmetry restored: $(-)^*$ is also the functor $\sum_{n \geq 0} \Set(\Fin\, n, -)$!
\end{example}

\begin{definition}
	A \textbf{doctrine of behaviour} for the doctrine $\dblcat{Doctrine}$ is map of doctrines of systems into the behavioural doctrine $\BSys$:
	\begin{equation}
		% https://q.uiver.app/?q=WzAsNSxbMCwwLCJcXGRibGNhdHtEYXRhfSJdLFswLDIsIlxcZGJsY2F0e0NhcnRDYXR9Il0sWzIsMSwiXFxUaGVvcmllcyJdLFsxLDBdLFsxLDJdLFswLDEsIlxcZGJsY2F0IEJeXFx0b3AiLDJdLFswLDIsIlxcZGJsY2F0e0RvY3RyaW5lfSIsMCx7ImN1cnZlIjotMX1dLFsxLDIsIlxcT2JzIiwyLHsiY3VydmUiOjF9XSxbMyw0LCJcXGRibGNhdCBCXlxcZmxhdCIsMCx7Im9mZnNldCI6NSwic2hvcnRlbiI6eyJzb3VyY2UiOjMwLCJ0YXJnZXQiOjMwfSwibGV2ZWwiOjJ9XV0=
		\begin{tikzcd}[ampersand replacement=\&, sep=small]
			{\dblcat{Data}} \& {} \\
			\&\& \Theories \\
			{\dblcat{CartCat}} \& {}
			\arrow["{\dblcat B^\top}"', from=1-1, to=3-1]
			\arrow["{\dblcat{Doctrine}}", curve={height=-6pt}, from=1-1, to=2-3]
			\arrow["\BSys"', curve={height=6pt}, from=3-1, to=2-3]
			\arrow["{\dblcat B^\flat}"', shift right=3, shorten <=10pt, shorten >=14pt, Rightarrow, from=1-2, to=3-2]
		\end{tikzcd}
	\end{equation}
\end{definition}

\begin{remark}
	The laxity of $B^\flat$ relates the behaviours of the parts of a system to the behaviour of the whole system. The non-invertibility of such a map witnesses emergent behaviours.
	In \cite[Theorem 5.3.3.1]{myers_categorical_2022}, Myers proves that a large class of behaviours for the doctrine of Moore machines does not, in fact, exhibit emergence, by showing such the laxity of $B^\flat$ is invertible.
\end{remark}

\begin{example}
	A large class of behaviours are corepresentables, i.e.~defined by simulations of an archetypal system exhibiting that behaviour.
	Thus there is a \textbf{doctrine of corepresentable behaviour} on the doctrine of pointed theories:
	\begin{equation}
		% https://q.uiver.app/?q=WzAsNSxbMCwwLCJcXFRoZW9yaWVzX1xcYXN0Il0sWzAsMiwiXFxkYmxjYXR7Q2FydENhdH0iXSxbMiwxLCJcXFRoZW9yaWVzIl0sWzEsMF0sWzEsMl0sWzAsMSwiXFxTZXQiLDJdLFswLDIsIlUiLDAseyJjdXJ2ZSI6LTF9XSxbMSwyLCJcXE9icyIsMix7ImN1cnZlIjoxfV0sWzMsNCwiXFxIb21eaCIsMix7Im9mZnNldCI6Mywic2hvcnRlbiI6eyJzb3VyY2UiOjMwLCJ0YXJnZXQiOjMwfSwibGV2ZWwiOjJ9XV0=
		\begin{tikzcd}[ampersand replacement=\&, sep=small]
			{\Theories_\ast} \& {} \\
			\&\& \Theories \\
			{\dblcat{CartCat}} \& {}
			\arrow["\Set"', from=1-1, to=3-1]
			\arrow["U", curve={height=-6pt}, from=1-1, to=2-3]
			\arrow["\BSys"', curve={height=6pt}, from=3-1, to=2-3]
			\arrow["{\Hom^h}"', shift right=3, shorten <=10pt, shorten >=14pt, Rightarrow, from=1-2, to=3-2]
		\end{tikzcd}
	\end{equation}
	An object of $\Theories_\ast$ is a systems theory equipped with a distinguished system (hence a pair of an interface and a system over it) which is used as archetype for a certain kind of behaviour.

	The transformation $\Hom^h$ at a pointed theory $(\Sys : \Processes^\top \to \dblCat, \sys B : \Sys(I))$ is the map of system theories $\Processes^h(\sys B, -) : \Sys \to \BSys(\Set)$ defined as follows.
	Its two components are the horizontal hom-functor $\Processes^h(I, -) : \Processes \to \dblSet$ and the similar fiberwise hom-functor
	\begin{equation}
		\Processes^h(\sys B, -)^\flat : \Sys \twoto \Set/\Processes^h(I, -).
	\end{equation}
	This latter functor sends a system $\sys S : \Sys(J)$ to the $\Processes^h(I, J)$-indexed family of sets sending a map of interfaces $k : I \to J$ to the set $\Sys(k)(\sys B, \sys S)$ of maps of systems mediated by $k$.
\end{example}

\begin{example}
	Let $\dblcat{MooreData}^!$ be the 2-category of fibrations of categories with a terminal object and a section thereof. These amount to a fibration $p:\cat E \to \cat C$ such that $p(1_{\cat E}) = 1_{\cat C}$, and a section $T : \cat C \to \cat E$ such that $T(1_{\cat C}) = 1_{\cat E}$.
	In this situation, one can build the Moore machine $\fix : \lens{T1}{1} \equalto \lens{1}{1}$ which `does nothing'. Hence we get a \emph{doctrine of fixpoints} by using such a machine as the archetype for the behaviour of a still system:
	\begin{equation}
		% https://q.uiver.app/?q=WzAsNixbMCwxLCJcXFRoZW9yaWVzX1xcYXN0Il0sWzAsMywiXFxkYmxjYXR7Q2FydENhdH0iXSxbMiwyLCJcXFRoZW9yaWVzIl0sWzEsMV0sWzEsM10sWzAsMCwiXFxkYmxjYXR7TW9vcmVEYXRhfV4xIl0sWzAsMSwiXFxTZXQiLDJdLFswLDIsIlUiLDAseyJjdXJ2ZSI6LTF9XSxbMSwyLCJcXE9icyIsMix7ImN1cnZlIjoxfV0sWzMsNCwiXFxIb21eaCIsMix7Im9mZnNldCI6Mywic2hvcnRlbiI6eyJzb3VyY2UiOjMwLCJ0YXJnZXQiOjMwfSwibGV2ZWwiOjJ9XSxbNSwyLCJcXE1vb3JlIiwwLHsiY3VydmUiOi0zfV0sWzUsMCwiKFxcTW9vcmUsIFxcZml4IDogXFxNb29yZSgxKSkiLDJdXQ==
		\begin{tikzcd}[ampersand replacement=\&, sep=small]
			{\dblcat{MooreData}^!} \\[2ex]
			{\Theories_\ast} \& {} \\
			\&\& \Theories \\
			{\dblcat{CartCat}} \& {}
			\arrow["\Set"', from=2-1, to=4-1]
			\arrow["U", curve={height=-6pt}, from=2-1, to=3-3]
			\arrow["\BSys"', curve={height=6pt}, from=4-1, to=3-3]
			\arrow["{\Hom^h}"', shift right=3, shorten <=10pt, shorten >=14pt, Rightarrow, from=2-2, to=4-2]
			\arrow["\Moore", curve={height=-18pt}, from=1-1, to=3-3]
			\arrow["{(\Moore,\ \fix : \Moore(1))}"', from=1-1, to=2-1]
		\end{tikzcd}
	\end{equation}
\end{example}

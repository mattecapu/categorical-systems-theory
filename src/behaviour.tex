\section{Behaviours}
\label{sec:behaviour}
Behaviours in CST are simply maps $\systh B : \Sys \to \BehSys[\cat C]$ into a behavioural theory.
By default, we consider behaviours valued in $\BehSys(\Set)$, but one can consider `structured behaviours' into $\BehSys[\cat C]$ when one wants to keep track of extra structure on the set of behaviours of a system.

The intuition behind this definition of behaviours the following definition is the following. First of all, $B$ maps interfaces to objects of observations we can make about them.
Every theory of behaviour gives its own notion of what `observation' means: it could be mere points, it could be distributions, it could be sequences of observations, and so on (examples will prove the variety this definition allows).
Then, we map every system to an object of states (which are \emph{observations we can make about the ways the system can be}) and a map that displays the dependency of said state on the observations on its interface (thus \emph{observations we can make about the ways the system can change}).

But mapping interfaces and systems to observations it's only half of what a behaviour does. The rest of the data (called below vertical and horizontal naturality) witness the \emph{compositionality} of such assignment. We say how maps of systems translate to maps between their behaviours and, most importantly, we say how composing systems and composing behaviours interact. The latter is thus a direct witness of \emph{emergence} of behaviour.

\begin{definition}
	A \textbf{theory of behaviour} or just \textbf{behaviour} for a systems theory $\Sys$ is a map of system theories $B:\Sys \to \BehSys[\cat E]$:
	\begin{equation}
		% https://q.uiver.app/?q=WzAsNSxbMCwwLCJcXFByb2Nlc3Nlc15cXHRvcCJdLFswLDIsIlxcU3BhbihcXGNhdCBDKV5cXHRvcCJdLFsyLDEsIlxcZGJsQ2F0Il0sWzEsMF0sWzEsMl0sWzAsMSwiQl5cXHRvcCIsMl0sWzAsMiwiXFxTeXMiLDAseyJjdXJ2ZSI6LTF9XSxbMSwyLCJcXGNhdCBDLy0iLDIseyJjdXJ2ZSI6MX1dLFszLDQsIkJeXFxmbGF0IiwwLHsib2Zmc2V0Ijo1LCJzaG9ydGVuIjp7InNvdXJjZSI6MzAsInRhcmdldCI6MzB9LCJsZXZlbCI6Mn1dXQ==
		\begin{tikzcd}[ampersand replacement=\&, sep=small]
			{\Comp^\top} \& {} \\
			\&\& \dblCat \\
			{\Span(\cat E)^\top} \& {}
			\arrow["{B^\top}"', from=1-1, to=3-1]
			\arrow["\Sys", curve={height=-6pt}, from=1-1, to=2-3]
			\arrow["{\cat E/-}"', curve={height=6pt}, from=3-1, to=2-3]
			\arrow["{B^\flat}"', shift right=4, shorten <=10pt, shorten >=15pt, Rightarrow, from=1-2, to=3-2]
		\end{tikzcd}
	\end{equation}
	Hence $B$ is given by
	\begin{enumerate}
		\item a unitary lax double functor
		\begin{equation}
			B : \Comp \unilaxto \Span(\cat E)
		\end{equation}
		\item a vertical lax-natural transformation whose components are given, for every $I: \Comp$, by
		\begin{equation}
			B^\flat_I : \Sys(I) \longto \cat E/B(I).
		\end{equation}
		\item a laxator witnessing vertical naturality for every process $p : I \verto J$ in $\Comp$:
		\begin{equation}
		\label{eq:beh-ver-laxator}
			% https://q.uiver.app/?q=WzAsNCxbMCwwLCJcXFN5cyhJKSJdLFswLDEsIlxcU3lzKEopIl0sWzEsMCwiXFxjYXQgQy9JIl0sWzEsMSwiXFxjYXQgQy9KIl0sWzAsMSwiXFxTeXMocCkiLDJdLFswLDIsIkJeXFxmbGF0X0kiXSxbMSwzLCJCXlxcZmxhdF9KIiwyXSxbMiwzLCJcXGNhdCBDL0IocCkiXSxbNiw3LCJCXlxcZmxhdF9wIiwwLHsib2Zmc2V0IjotMiwic2hvcnRlbiI6eyJzb3VyY2UiOjIwLCJ0YXJnZXQiOjIwfX1dXQ==
			\begin{tikzcd}[ampersand replacement=\&]
				{\Sys(I)} \& {\cat E/B(I)} \\
				{\Sys(J)} \& {\cat E/B(J)}
				\arrow["{\Sys(p)}"', from=1-1, to=2-1]
				\arrow["{B^\flat_I}", from=1-1, to=1-2]
				\arrow[""{name=0, anchor=center, inner sep=0}, "{B^\flat_J}"', from=2-1, to=2-2]
				\arrow[""{name=1, anchor=center, inner sep=0}, "{\cat E/B(p)}", from=1-2, to=2-2]
				\arrow["{B^\flat_p}", shift left=2, shorten <=5pt, shorten >=5pt, Rightarrow, from=0, to=1]
			\end{tikzcd},
			\ \text{or alternatively:}
			% https://q.uiver.app/?q=WzAsNixbMCwwLCJcXFN5cyhJKSJdLFswLDEsIlxcU3lzKEopIl0sWzEsMSwiXFxjYXQgQy9JIl0sWzAsMiwiXFxjYXQgQy9KIl0sWzEsMiwiXFxjYXQgQy9KIl0sWzEsMCwiXFxTeXMoSSkiXSxbMCwxLCJcXFN5cyhwKSIsMl0sWzEsMywiQl5cXGZsYXRfSiIsMl0sWzIsNCwiXFxjYXQgQy9CKHApIl0sWzUsMiwiQl5cXGZsYXRfSSJdLFswLDUsIiIsMCx7ImxldmVsIjoyLCJzdHlsZSI6eyJib2R5Ijp7Im5hbWUiOiJiYXJyZWQifSwiaGVhZCI6eyJuYW1lIjoibm9uZSJ9fX1dLFszLDQsIiIsMix7ImxldmVsIjoyLCJzdHlsZSI6eyJib2R5Ijp7Im5hbWUiOiJiYXJyZWQifSwiaGVhZCI6eyJuYW1lIjoibm9uZSJ9fX1dLFsxMCwxMSwiQl5cXGZsYXRfcCIsMSx7InNob3J0ZW4iOnsic291cmNlIjoyMCwidGFyZ2V0IjoyMH19XV0=
			\begin{tikzcd}[ampersand replacement=\&]
				{\Sys(I)} \& {\Sys(I)} \\
				{\Sys(J)} \& {\cat E/I} \\
				{\cat E/J} \& {\cat E/J}
				\arrow["{\Sys(p)}"', from=1-1, to=2-1]
				\arrow["{B^\flat_J}"', from=2-1, to=3-1]
				\arrow["{\cat E/B(p)}", from=2-2, to=3-2]
				\arrow["{B^\flat_I}", from=1-2, to=2-2]
				\arrow[""{name=0, anchor=center, inner sep=0}, "\shortmid"{marking}, Rightarrow, no head, from=1-1, to=1-2]
				\arrow[""{name=1, anchor=center, inner sep=0}, "\shortmid"{marking}, Rightarrow, no head, from=3-1, to=3-2]
				\arrow["{B^\flat_p}"{description}, shorten <=9pt, shorten >=9pt, Rightarrow, from=0, to=1]
			\end{tikzcd}
		\end{equation}
		which (when $p$ represents a coupling of some sort, hence a composition of systems) sends a behaviour of the composite to a composite of behaviours,
		\item a square in $\dblCat$ witnessing horizontal naturality for every map of interfaces $h : I \to J$ in $\Comp$:
		\begin{equation}
			% https://q.uiver.app/?q=WzAsNCxbMCwwLCJcXFN5cyhJKSJdLFswLDEsIlxcU3lzKEspIl0sWzEsMCwiXFxjYXQgQy9JIl0sWzEsMSwiXFxjYXQgQy9KIl0sWzAsMSwiXFxTeXMoaCkiLDIseyJzdHlsZSI6eyJib2R5Ijp7Im5hbWUiOiJiYXJyZWQifX19XSxbMiwzLCJcXGNhdCBDL0IoaCkiLDAseyJzdHlsZSI6eyJib2R5Ijp7Im5hbWUiOiJiYXJyZWQifX19XSxbMCwyLCJCXlxcZmxhdF9JIl0sWzEsMywiQl5cXGZsYXRfSiIsMl0sWzQsNSwiQl5cXGZsYXRfaCIsMSx7InNob3J0ZW4iOnsic291cmNlIjoyMCwidGFyZ2V0IjoyMH19XV0=
			\begin{tikzcd}[ampersand replacement=\&]
				{\Sys(I)} \& {\cat E/B(I)} \\
				{\Sys(J)} \& {\cat E/B(J)}
				\arrow[""{name=0, anchor=center, inner sep=0}, "{\Sys(h)}"', "\shortmid"{marking}, from=1-1, to=2-1]
				\arrow[""{name=1, anchor=center, inner sep=0}, "{\cat E/B(h)}", "\shortmid"{marking}, from=1-2, to=2-2]
				\arrow["{B^\flat_I}", from=1-1, to=1-2]
				\arrow["{B^\flat_J}"', from=2-1, to=2-2]
				\arrow["{B^\flat_h}"{description}, shorten <=9pt, shorten >=9pt, Rightarrow, from=0, to=1]
			\end{tikzcd}
		\end{equation}
		which sends every map of systems (over $h$) to a map of their behaviours (over $B(h)$)
	\end{enumerate}
	This data satisfies the following coherences:
	\begin{enumerate}
		\item The laxators witnessing vertical functoriality compose; hence for each composable pair of processes $I \nverto{p} J \nverto{q} K$ in $\Comp$, we have:
		\begin{equation}
			% https://q.uiver.app/?q=WzAsNixbMCwwLCJcXFN5cyhJKSJdLFswLDEsIlxcU3lzKEopIl0sWzEsMCwiXFxjYXQgQy9JIl0sWzEsMSwiXFxjYXQgQy9KIl0sWzAsMiwiXFxTeXMoSykiXSxbMSwyLCJcXGNhdCBDL0siXSxbMCwxLCJcXFN5cyhwKSIsMl0sWzAsMiwiQl5cXGZsYXRfSSJdLFsxLDMsIkJeXFxmbGF0X0oiLDJdLFsyLDMsIlxcY2F0IEMvQihwKSJdLFsxLDQsIlxcU3lzKHEpIiwyXSxbMyw1LCJcXGNhdCBDL0IocSkiXSxbNCw1LCJCXlxcZmxhdF9LIiwyXSxbOCw5LCJCXlxcZmxhdF9wIiwwLHsib2Zmc2V0IjotMiwic2hvcnRlbiI6eyJzb3VyY2UiOjIwLCJ0YXJnZXQiOjIwfX1dLFsxMiwxMSwiQl5cXGZsYXRfcSIsMCx7Im9mZnNldCI6LTEsInNob3J0ZW4iOnsic291cmNlIjoyMCwidGFyZ2V0IjoyMH19XV0=
			\begin{tikzcd}[ampersand replacement=\&]
				{\Sys(I)} \& {\cat E/B(I)} \\
				{\Sys(J)} \& {\cat E/B(J)} \\
				{\Sys(K)} \& {\cat E/B(K)}
				\arrow["{\Sys(p)}"', from=1-1, to=2-1]
				\arrow["{B^\flat_I}", from=1-1, to=1-2]
				\arrow[""{name=0, anchor=center, inner sep=0}, "{B^\flat_J}"', from=2-1, to=2-2]
				\arrow[""{name=1, anchor=center, inner sep=0}, "{\cat E/B(p)}", from=1-2, to=2-2]
				\arrow["{\Sys(q)}"', from=2-1, to=3-1]
				\arrow[""{name=2, anchor=center, inner sep=0}, "{\cat E/B(q)}", from=2-2, to=3-2]
				\arrow[""{name=3, anchor=center, inner sep=0}, "{B^\flat_K}"', from=3-1, to=3-2]
				\arrow["{B^\flat_p}", shift left=2, shorten <=5pt, shorten >=5pt, Rightarrow, from=0, to=1]
				\arrow["{B^\flat_q}", shift left=1, shorten <=5pt, shorten >=5pt, Rightarrow, from=3, to=2]
			\end{tikzcd}
			=
			% https://q.uiver.app/?q=WzAsNixbMCwwLCJcXFN5cyhJKSJdLFswLDEsIlxcU3lzKEopIl0sWzEsMCwiXFxjYXQgQy9JIl0sWzEsMSwiXFxjYXQgQy9KIl0sWzAsMiwiXFxTeXMoSykiXSxbMSwyLCJcXGNhdCBDL0siXSxbMCwxLCJcXFN5cyhwKSIsMl0sWzAsMiwiQl5cXGZsYXRfSSJdLFsyLDMsIlxcY2F0IEMvQihwKSJdLFsxLDQsIlxcU3lzKHEpIiwyXSxbMyw1LCJcXGNhdCBDL0IocSkiXSxbNCw1LCJCXlxcZmxhdF9LIiwyXSxbMTEsOCwiQl5cXGZsYXRfe3AgXFxjb21wIHF9IiwwLHsib2Zmc2V0IjotMSwic2hvcnRlbiI6eyJzb3VyY2UiOjIwLCJ0YXJnZXQiOjIwfX1dXQ==
			\begin{tikzcd}[ampersand replacement=\&]
				{\Sys(I)} \& {\cat E/B(I)} \\
				{\Sys(J)} \& {\cat E/B(J)} \\
				{\Sys(K)} \& {\cat E/B(K)}
				\arrow["{\Sys(p)}"', from=1-1, to=2-1]
				\arrow["{B^\flat_I}", from=1-1, to=1-2]
				\arrow[""{name=0, anchor=center, inner sep=0}, "{\cat E/B(p)}", from=1-2, to=2-2]
				\arrow["{\Sys(q)}"', from=2-1, to=3-1]
				\arrow["{\cat E/B(q)}", from=2-2, to=3-2]
				\arrow[""{name=1, anchor=center, inner sep=0}, "{B^\flat_K}"', from=3-1, to=3-2]
				\arrow["{B^\flat_{p \comp q}}", shift left=4, shorten <=8pt, shorten >=8pt, Rightarrow, from=1, to=0]
			\end{tikzcd}
		\end{equation}
		Moreover, $B(1) = 1$.
		\item The squares witnessing horizontal naturality commute with the laxators of $\Sys$ and $B^*\cat E/-$ (the composite of $B$ and $\cat E/-$); hence for each composable pair of maps of interfaces $I \nhorto{h} J \nhorto{k} K$ in $\Comp$, we have:
		\begin{equation}
			% https://q.uiver.app/?q=WzAsOCxbMSwwLCJcXFN5cyhJKSJdLFsxLDEsIlxcU3lzKEopIl0sWzIsMCwiXFxjYXQgQy9JIl0sWzIsMSwiXFxjYXQgQy9KIl0sWzEsMiwiXFxTeXMoSykiXSxbMiwyLCJcXGNhdCBDL0siXSxbMCwwLCJcXFN5cyhJKSJdLFswLDIsIlxcU3lzKEspIl0sWzAsMSwiXFxTeXMoaCkiLDIseyJzdHlsZSI6eyJib2R5Ijp7Im5hbWUiOiJiYXJyZWQifX19XSxbMiwzLCJcXGNhdCBDL0IoaCkiLDAseyJzdHlsZSI6eyJib2R5Ijp7Im5hbWUiOiJiYXJyZWQifX19XSxbMCwyLCJCXlxcZmxhdF9JIl0sWzEsMywiQl5cXGZsYXRfSiIsMl0sWzEsNCwiXFxTeXMoaykiLDIseyJzdHlsZSI6eyJib2R5Ijp7Im5hbWUiOiJiYXJyZWQifX19XSxbMyw1LCJcXGNhdCBDL0IoaykiLDAseyJzdHlsZSI6eyJib2R5Ijp7Im5hbWUiOiJiYXJyZWQifX19XSxbNCw1LCJCXlxcZmxhdF9LIiwyXSxbNiw3LCJcXFN5cyhoIFxcY29tcCBrKSIsMix7InN0eWxlIjp7ImJvZHkiOnsibmFtZSI6ImJhcnJlZCJ9fX1dLFs2LDAsIiIsMix7ImxldmVsIjoyLCJzdHlsZSI6eyJoZWFkIjp7Im5hbWUiOiJub25lIn19fV0sWzcsNCwiIiwyLHsibGV2ZWwiOjIsInN0eWxlIjp7ImhlYWQiOnsibmFtZSI6Im5vbmUifX19XSxbOCw5LCJCXlxcZmxhdF9oIiwxLHsic2hvcnRlbiI6eyJzb3VyY2UiOjIwLCJ0YXJnZXQiOjIwfX1dLFsxMiwxMywiQl5cXGZsYXRfayIsMSx7InNob3J0ZW4iOnsic291cmNlIjoyMCwidGFyZ2V0IjoyMH19XSxbMTUsMSwiXFxlbGxeXFxTeXNfe2gsa30iLDEseyJzaG9ydGVuIjp7InNvdXJjZSI6MjB9fV1d
			\begin{tikzcd}[ampersand replacement=\&]
				{\Sys(I)} \&[-3ex] {\Sys(I)} \&[-3ex] {\cat E/B(I)} \\
				\& {\Sys(J)} \& {\cat E/B(J)} \\
				{\Sys(K)} \& {\Sys(K)} \& {\cat E/B(K)}
				\arrow[""{name=0, anchor=center, inner sep=0}, "{\Sys(h)}"', "\shortmid"{marking}, from=1-2, to=2-2]
				\arrow[""{name=1, anchor=center, inner sep=0}, "{\cat E/B(h)}", "\shortmid"{marking}, from=1-3, to=2-3]
				\arrow["{B^\flat_I}", from=1-2, to=1-3]
				\arrow["{B^\flat_J}"', from=2-2, to=2-3]
				\arrow[""{name=2, anchor=center, inner sep=0}, "{\Sys(k)}"', "\shortmid"{marking}, from=2-2, to=3-2]
				\arrow[""{name=3, anchor=center, inner sep=0}, "{\cat E/B(k)}", "\shortmid"{marking}, from=2-3, to=3-3]
				\arrow["{B^\flat_K}"', from=3-2, to=3-3]
				\arrow[""{name=4, anchor=center, inner sep=0}, "{\Sys(h \comp k)}"{description, pos=0.3}, "\shortmid"{marking}, from=1-1, to=3-1]
				\arrow[Rightarrow, no head, from=1-1, to=1-2]
				\arrow[Rightarrow, no head, from=3-1, to=3-2]
				\arrow["{B^\flat_h}"{description}, shorten <=9pt, shorten >=9pt, Rightarrow, from=0, to=1]
				\arrow["{B^\flat_k}"{description}, shorten <=9pt, shorten >=9pt, Rightarrow, from=2, to=3]
				\arrow["{\ell^\Sys_{h,k}}"{description}, shorten <=6pt, Rightarrow, from=4, to=2-2]
			\end{tikzcd}
			=
			\begin{tikzcd}[ampersand replacement=\&]
				{\Sys(I)} \&[-3ex] {\cat E/B(I)} \& {\cat E/B(I)} \\
				{\Sys(J)} \& {\cat E/B(J)} \\
				{\Sys(K)} \& {\cat E/B(K)} \& {\cat E/B(K)}
				\arrow[""{name=0, anchor=center, inner sep=0}, "{\Sys(h)}"', "\shortmid"{marking}, from=1-1, to=2-1]
				\arrow[""{name=1, anchor=center, inner sep=0}, "{\cat E/B(h)}", "\shortmid"{marking}, from=1-2, to=2-2]
				\arrow["{B^\flat_I}", from=1-1, to=1-2]
				\arrow["{B^\flat_J}"', from=2-1, to=2-2]
				\arrow[""{name=2, anchor=center, inner sep=0}, "{\Sys(k)}"', "\shortmid"{marking}, from=2-1, to=3-1]
				\arrow[""{name=3, anchor=center, inner sep=0}, "{\cat E/B(k)}", "\shortmid"{marking}, from=2-2, to=3-2]
				\arrow["{B^\flat_K}"', from=3-1, to=3-2]
				\arrow[""{name=4, anchor=center, inner sep=0}, "{\cat E/(B(h \comp k))}"{description, pos=0.3}, "\shortmid"{marking}, from=1-3, to=3-3]
				\arrow[Rightarrow, no head, from=3-2, to=3-3]
				\arrow[Rightarrow, no head, from=1-2, to=1-3]
				\arrow["{B^\flat_h}"{description}, shorten <=9pt, shorten >=9pt, Rightarrow, from=0, to=1]
				\arrow["{B^\flat_k}"{description}, shorten <=9pt, shorten >=9pt, Rightarrow, from=2, to=3]
				\arrow["{\ell^{B^*\cat E/-}_{h,k}}"{description}, shorten >=6pt, Rightarrow, from=2-2, to=4]
			\end{tikzcd}
		\end{equation}
		Notice the laxator of $B^*\cat E/-$ is given by
		\begin{equation}
			\cat E/B(h) \comp \cat E/B(k) \nlongto{\ell^{\cat E/-}_{B(h),B(k)}} \cat E/(B(h)\comp B(k)) \nlongto{\cat E/\ell^B_{h,k}} \cat E/B(h \comp k).
		\end{equation}
		Moreover, $B(1) = 1$.
		\item For every square $\begin{tikzcd}[ampersand replacement=\&, sep=small]
			I \&[1ex] J \\[1ex]
			{I'} \& {J'}
			\arrow[""{name=0, anchor=center, inner sep=0}, "p"', "\bullet"{marking}, from=1-1, to=2-1]
			\arrow[""{name=1, anchor=center, inner sep=0}, "{p'}", "\bullet"{marking}, from=1-2, to=2-2]
			\arrow["{h'}"', from=2-1, to=2-2]
			\arrow["h", from=1-1, to=1-2]
			\arrow["\alpha"{description}, shorten <=6pt, shorten >=6pt, Rightarrow, from=0, to=1]
		\end{tikzcd}$ in $\Comp$, the following `cube' commutes:
		\begin{eqalign}
			% https://q.uiver.app/?q=WzAsOSxbMCwwLCJcXFN5cyhJKSJdLFsyLDAsIlxcU3lzKEopIl0sWzAsMiwiXFxTeXMoSScpIl0sWzIsMiwiXFxTeXMoSicpIl0sWzQsMiwiXFxjYXQgQy9CKEopIl0sWzAsNCwiXFxjYXQgQy9CKEknKSJdLFsyLDQsIlxcY2F0IEMvQihKJykiXSxbNCw0LCJcXGNhdCBDL0IoSicpIl0sWzQsMCwiXFxTeXMoSikiXSxbMCwyLCJcXFN5cyhwKSIsMV0sWzEsMywiXFxTeXMocCcpIiwxXSxbMiwzLCJcXFN5cyhoJykiLDIseyJzdHlsZSI6eyJib2R5Ijp7Im5hbWUiOiJiYXJyZWQifX19XSxbMCwxLCJcXFN5cyhoKSIsMCx7InN0eWxlIjp7ImJvZHkiOnsibmFtZSI6ImJhcnJlZCJ9fX1dLFs1LDYsIlxcY2F0IEMvQihoJykiLDIseyJzdHlsZSI6eyJib2R5Ijp7Im5hbWUiOiJiYXJyZWQifX19XSxbMiw1LCJCXlxcZmxhdF97SSd9IiwxXSxbMyw2LCJCXlxcZmxhdF97Sid9IiwxXSxbNCw3LCJcXGNhdCBDL0IocCcpIiwxXSxbOCw0LCJCXlxcZmxhdF9KIiwxXSxbMSw4LCIiLDEseyJsZXZlbCI6Miwic3R5bGUiOnsiYm9keSI6eyJuYW1lIjoiYmFycmVkIn0sImhlYWQiOnsibmFtZSI6Im5vbmUifX19XSxbNiw3LCIiLDEseyJsZXZlbCI6Miwic3R5bGUiOnsiYm9keSI6eyJuYW1lIjoiYmFycmVkIn0sImhlYWQiOnsibmFtZSI6Im5vbmUifX19XSxbMTIsMTEsIlxcU3lzKFxcYWxwaGEpIiwxLHsic2hvcnRlbiI6eyJzb3VyY2UiOjIwLCJ0YXJnZXQiOjIwfX1dLFsxMSwxMywiQl5cXGZsYXRfe2gnfSIsMSx7InNob3J0ZW4iOnsic291cmNlIjoyMCwidGFyZ2V0IjoyMH19XSxbMTgsMTksIkJeXFxmbGF0X3twJ30iLDEseyJzaG9ydGVuIjp7InNvdXJjZSI6MjAsInRhcmdldCI6MjB9fV1d
			\begin{tikzcd}[ampersand replacement=\&,sep=scriptsize]
				{\Sys(I)} \&\& {\Sys(J)} \&[-1ex]\&[-1ex] {\Sys(J)} \\
				\\
				{\Sys(I')} \&\& {\Sys(J')} \&\& {\cat E/B(J)} \\
				\\
				{\cat E/B(I')} \&\& {\cat E/B(J')} \&\& {\cat E/B(J')}
				\arrow["{\Sys(p)}"{description}, from=1-1, to=3-1]
				\arrow["{\Sys(p')}"{description}, from=1-3, to=3-3]
				\arrow[""{name=0, anchor=center, inner sep=0}, "{\Sys(h')}"', "\shortmid"{marking}, from=3-1, to=3-3]
				\arrow[""{name=1, anchor=center, inner sep=0}, "{\Sys(h)}", "\shortmid"{marking}, from=1-1, to=1-3]
				\arrow[""{name=2, anchor=center, inner sep=0}, "{\cat E/B(h')}"', "\shortmid"{marking}, from=5-1, to=5-3]
				\arrow["{B^\flat_{I'}}"{description}, from=3-1, to=5-1]
				\arrow["{B^\flat_{J'}}"{description}, from=3-3, to=5-3]
				\arrow["{\cat E/B(p')}"{description}, from=3-5, to=5-5]
				\arrow["{B^\flat_J}"{description}, from=1-5, to=3-5]
				\arrow[""{name=3, anchor=center, inner sep=0}, "\shortmid"{marking}, Rightarrow, no head, from=1-3, to=1-5]
				\arrow[""{name=4, anchor=center, inner sep=0}, "\shortmid"{marking}, Rightarrow, no head, from=5-3, to=5-5]
				\arrow["{\Sys(\alpha)}"{description}, shorten <=9pt, shorten >=9pt, Rightarrow, from=1, to=0]
				\arrow["{B^\flat_{h'}}"{description}, shorten <=12pt, shorten >=9pt, Rightarrow, from=0, to=2]
				\arrow["{B^\flat_{p'}}"{description}, shorten <=17pt, shorten >=17pt, Rightarrow, from=3, to=4]
			\end{tikzcd}
			=
			% https://q.uiver.app/?q=WzAsOSxbMiwwLCJcXFN5cyhJKSJdLFs0LDAsIlxcU3lzKEopIl0sWzAsMiwiXFxTeXMoSScpIl0sWzIsMiwiXFxjYXQgQy9CKEkpIl0sWzQsMiwiXFxjYXQgQy9CKEopIl0sWzIsNCwiXFxjYXQgQy9CKEknKSJdLFs0LDQsIlxcY2F0IEMvQihKJykiXSxbMCwwLCJcXFN5cyhJKSJdLFswLDQsIlxcY2F0IEMvQihJJykiXSxbMCwxLCJcXFN5cyhoKSIsMix7InN0eWxlIjp7ImJvZHkiOnsibmFtZSI6ImJhcnJlZCJ9fX1dLFszLDQsIlxcY2F0IEMvQihoKSIsMix7InN0eWxlIjp7ImJvZHkiOnsibmFtZSI6ImJhcnJlZCJ9fX1dLFs1LDYsIlxcY2F0IEMvQihoJykiLDIseyJzdHlsZSI6eyJib2R5Ijp7Im5hbWUiOiJiYXJyZWQifX19XSxbMyw1LCJcXGNhdCBDL0IocCkiLDFdLFs0LDYsIlxcY2F0IEMvQihwJykiLDFdLFswLDMsIkJeXFxmbGF0X0kiLDFdLFsxLDQsIkJeXFxmbGF0X0oiLDFdLFsyLDgsIkJeXFxmbGF0X3tJJ30iLDFdLFs3LDIsIlxcU3lzKHApIiwxXSxbNywwLCIiLDEseyJsZXZlbCI6Miwic3R5bGUiOnsiYm9keSI6eyJuYW1lIjoiYmFycmVkIn0sImhlYWQiOnsibmFtZSI6Im5vbmUifX19XSxbOCw1LCIiLDEseyJsZXZlbCI6Miwic3R5bGUiOnsiYm9keSI6eyJuYW1lIjoiYmFycmVkIn0sImhlYWQiOnsibmFtZSI6Im5vbmUifX19XSxbMTAsMTEsIlxcY2F0IEMvQihcXGFscGhhKSIsMSx7InNob3J0ZW4iOnsic291cmNlIjoyMCwidGFyZ2V0IjoyMH19XSxbOSwxMCwiQl5cXGZsYXRfaCIsMSx7InNob3J0ZW4iOnsic291cmNlIjoyMCwidGFyZ2V0IjoyMH19XSxbMTgsMTksIkJeXFxmbGF0X3AiLDEseyJzaG9ydGVuIjp7InNvdXJjZSI6MjAsInRhcmdldCI6MjB9fV1d
			\begin{tikzcd}[ampersand replacement=\&,sep=scriptsize]
				{\Sys(I)} \&\& {\Sys(I)} \&[-1ex]\&[-1ex] {\Sys(J)} \\
				\\
				{\Sys(I')} \&\& {\cat E/B(I)} \&\& {\cat E/B(J)} \\
				\\
				{\cat E/B(I')} \&\& {\cat E/B(I')} \&\& {\cat E/B(J')}
				\arrow[""{name=0, anchor=center, inner sep=0}, "{\Sys(h)}"', "\shortmid"{marking}, from=1-3, to=1-5]
				\arrow[""{name=1, anchor=center, inner sep=0}, "{\cat E/B(h)}"', "\shortmid"{marking}, from=3-3, to=3-5]
				\arrow[""{name=2, anchor=center, inner sep=0}, "{\cat E/B(h')}"', "\shortmid"{marking}, from=5-3, to=5-5]
				\arrow["{\cat E/B(p)}"{description}, from=3-3, to=5-3]
				\arrow["{\cat E/B(p')}"{description}, from=3-5, to=5-5]
				\arrow["{B^\flat_I}"{description}, from=1-3, to=3-3]
				\arrow["{B^\flat_J}"{description}, from=1-5, to=3-5]
				\arrow["{B^\flat_{I'}}"{description}, from=3-1, to=5-1]
				\arrow["{\Sys(p)}"{description}, from=1-1, to=3-1]
				\arrow[""{name=3, anchor=center, inner sep=0}, "\shortmid"{marking}, Rightarrow, no head, from=1-1, to=1-3]
				\arrow[""{name=4, anchor=center, inner sep=0}, "\shortmid"{marking}, Rightarrow, no head, from=5-1, to=5-3]
				\arrow["{\cat E/B(\alpha)}"{description}, shorten <=12pt, shorten >=9pt, Rightarrow, from=1, to=2]
				\arrow["{B^\flat_h}"{description}, shorten <=9pt, shorten >=9pt, Rightarrow, from=0, to=1]
				\arrow["{B^\flat_p}"{description}, shorten <=17pt, shorten >=17pt, Rightarrow, from=3, to=4]
			\end{tikzcd}
		\end{eqalign}
		Notice we used the `transposed' version of the vertical laxators of $B^\flat$ (diagram on the right of~\eqref{eq:beh-ver-laxator}).
	\end{enumerate}
\end{definition}

\begin{example}[States]
\label{ex:states}
	One of the simplest notion of behaviour for a system is given by the idea of `state space'. This is the very classical stance that what can be observed about a system is its permanence in a state which varies in a prescribed space.
	% So suppose $\cat C$ is some category of spaces\footnote{Following Lawvere~\cite{lawvere_categories_1992}, we consider `cartesian \& extensive' to be satisfying notion of \emph{category of spaces}.} and let's consider the theory of coalgebras $\Coalg_{\cat C} : \dblEnd^\top \unilaxto \dblCat$.

	% Then there is a theory of behaviour:
	% \begin{equation}
	% 	% https://q.uiver.app/?q=WzAsNSxbMCwwLCJcXGRibGNhdHtFbmR9XlxcdG9wIl0sWzAsMiwiXFxTcGFuKFxcY2F0IFMpXlxcdG9wIl0sWzIsMSwiXFxkYmxDYXQiXSxbMSwwXSxbMSwyXSxbMCwxLCIxXlxcdG9wIiwyXSxbMCwyLCJcXENvYWxnIiwwLHsiY3VydmUiOi0xfV0sWzEsMiwiXFxjYXQgUy8tIiwyLHsiY3VydmUiOjF9XSxbMyw0LCJcXHN0YXRlcyIsMix7ImxhYmVsX3Bvc2l0aW9uIjo0MCwib2Zmc2V0Ijo0LCJzaG9ydGVuIjp7InNvdXJjZSI6MzAsInRhcmdldCI6NTB9LCJsZXZlbCI6Mn1dXQ==
	% 	\begin{tikzcd}[ampersand replacement=\&,sep=small]
	% 		{\dblEnd^\top} \& {} \\
	% 		\&\& \dblCat \\
	% 		{\Span(\cat C)^\top} \& {}
	% 		\arrow["{1^\top}"', from=1-1, to=3-1]
	% 		\arrow["\Coalg_{\cat C}", curve={height=-6pt}, from=1-1, to=2-3]
	% 		\arrow["{\cat C/-}"', curve={height=6pt}, from=3-1, to=2-3]
	% 		\arrow["\states"'{pos=0.4}, shift right=4, shorten <=10pt, shorten >=15pt, Rightarrow, from=1-2, to=3-2]
	% 	\end{tikzcd}
	% \end{equation}
	% which is trivial on the interfaces and maps coalgebras to their carriers:
	% \begin{eqalign}
	% 	\states_F : \Coalg_{\cat C}(F) &\longto \cat C\\
	% 				(S, S \nto{\delta} FS) &\longmapsto S.
	% \end{eqalign}

	Clearly such a theory of behaviour can be defined for Moore machines. Let $\Moore_{(F, T)} : \dblLens^\top \unilaxto \dblCat$ be a theory of Moore machines, then there is again a theory of behaviour
	\begin{equation}
		% https://q.uiver.app/?q=WzAsNSxbMCwwLCJcXGRibExlbnNeXFx0b3AiXSxbMCwyLCJcXFNwYW4oXFxjYXQgQyleXFx0b3AiXSxbMiwxLCJcXGRibENhdCJdLFsxLDBdLFsxLDJdLFswLDEsIjFeXFx0b3AiLDJdLFswLDIsIlxcTW9vcmVfeyhGLCBUKX0iLDAseyJjdXJ2ZSI6LTF9XSxbMSwyLCJcXGNhdCBDLy0iLDIseyJjdXJ2ZSI6MX1dLFszLDQsIlxcc3RhdGVzIiwyLHsibGFiZWxfcG9zaXRpb24iOjQwLCJvZmZzZXQiOjQsInNob3J0ZW4iOnsic291cmNlIjozMCwidGFyZ2V0Ijo1MH0sImxldmVsIjoyfV1d
		\begin{tikzcd}[ampersand replacement=\&,sep=small]
			{\dblLens^\top} \& {} \\
			\&\& \dblCat \\
			{\Span(\cat C)^\top} \& {}
			\arrow["{1^\top}"', from=1-1, to=3-1]
			\arrow["{\Moore_{(F, T)}}", curve={height=-6pt}, from=1-1, to=2-3]
			\arrow["{\cat C/-}"', curve={height=6pt}, from=3-1, to=2-3]
			\arrow["\states"'{pos=0.4}, shift right=4, shorten <=10pt, shorten >=16pt, Rightarrow, from=1-2, to=3-2]
		\end{tikzcd}
	\end{equation}
	which is trivial on the interfaces and maps Moore machines to their state spaces:
	\begin{eqalign}
		\states_{\lens{I}{O}} : \Moore_{(F,T)}\lens{I}{O} &\longto \cat C\\
					(S, \lens{\update}{\expose} : \lens{TS}{S} \opticto \lens{I}{O}) &\longmapsto S.
	\end{eqalign}
\end{example}

\begin{example}[Paths]
\label{ex:paths}
	Labelled transition systems admit a behaviour $\systh P : \Trans \to \BehSys$ given by paths:
	\begin{equation}
		\begin{tikzcd}[ampersand replacement=\&,sep=small]
			{\Alph^\top} \& {} \\
			\&\& \dblCat \\
			{\Span(\Set)^\top} \& {}
			\arrow["{{(-)^*}^\top}"', from=1-1, to=3-1]
			\arrow["{\Trans}", curve={height=-6pt}, from=1-1, to=2-3]
			\arrow["{\Set/-}"', curve={height=6pt}, from=3-1, to=2-3]
			\arrow["P"'{pos=0.4}, shift right=4, shorten <=10pt, shorten >=16pt, Rightarrow, from=1-2, to=3-2]
		\end{tikzcd}
	\end{equation}
	where
	\begin{enumerate}
		\item $\Trans : \Alph^\top \unilaxto \dblCat$ is the theory of systems introduced in~\cref{ex:trans-sys};
		\item $(-)^* : \Alph \to \Span(\Set)$ sends a finite alphabet $\Sigma$ to the free monoid $\Sigma^*$, an alphabet mapping $h:\Sigma \to \Sigma'$ to the morphism of monoids $h^* : \Sigma^* \to {\Sigma'}^*$, an alphabet reduction $p: \Sigma \from \Xi$ to the span $\Sigma^* \nfrom{p^*} \Xi^* \equalto \Xi^*$, and a commutative square to an obvious morphism of spans;
		\item The component at $\Sigma : \Alph$ of $P$ is
		\begin{eqalign}
			P_\Sigma : \Trans(\Sigma) &\longto \Set/\Sigma^*\\
			S \times \Sigma \nto{\delta} S &\longmapsto \{s \nreachto{w}_\delta s', w \in \Sigma^* \} \nlongto{\pi_{\Sigma^*}} \Sigma^*
		\end{eqalign}
		where $s  \nreachto{w}_\delta s'$ denotes a triple $(w, s, s') \in \Sigma^* \times S \times S$ such that $s'$ can be reached from $s$ by following the transitions $w_1 \cdots w_n = w$ determined by the transition map $\delta$.
		\item The vertical lax natural structure of $P$ is given, for each alphabet reduction $p: \Sigma \from \Xi$, by a natural transformation $P_p$ filling the square:
		\begin{equation}
			% https://q.uiver.app/?q=WzAsNCxbMCwwLCJcXFRyYW5zKFxcU2lnbWEpIl0sWzAsMSwiXFxUcmFucyhcXFhpKSJdLFsxLDAsIlxcU2V0L1xcU2lnbWEiXSxbMSwxLCJcXFNldC9cXFhpIl0sWzIsMywiXFxTZXQvcF4qIl0sWzAsMiwiUl9cXFNpZ21hIl0sWzEsMywiUl9cXFhpIiwyXSxbMCwxLCJcXFRyYW5zKHApIiwyXSxbNiw0LCJSX3AiLDAseyJvZmZzZXQiOi0zLCJzaG9ydGVuIjp7InNvdXJjZSI6MjAsInRhcmdldCI6MjB9fV1d
			\begin{tikzcd}[ampersand replacement=\&]
				{\Trans(\Sigma)} \& {\Set/\Sigma} \\
				{\Trans(\Xi)} \& {\Set/\Xi}
				\arrow[""{name=0, anchor=center, inner sep=0}, "{\Set/p^*}", from=1-2, to=2-2]
				\arrow["{P_\Sigma}", from=1-1, to=1-2]
				\arrow[""{name=1, anchor=center, inner sep=0}, "{P_\Xi}"', from=2-1, to=2-2]
				\arrow["{\Trans(p)}"', from=1-1, to=2-1]
				\arrow["{P_p}", shift left=3, shorten <=4pt, shorten >=4pt, Rightarrow, from=1, to=0]
			\end{tikzcd}
		\end{equation}
		We define it as follows. Given $S \times \Sigma \nto{\delta} S : \Trans(\Sigma)$, one gets, following the bottom path:
		\begin{equation}
			S \times \Sigma \nto{\delta} S
			\quad\mapsto\quad
			S \times \Xi \nto{S \times p} S \times \Sigma \nto{\delta} S
			\quad\mapsto\quad
			\begin{tikzcd}[ampersand replacement=\&]
				{\{s \nreachto{v}_{\Trans(p)(\delta)} s', v \in \Xi^*\}} \\
				{\Xi^*}
				\arrow["{\pi_{\Xi^*}}", from=1-1, to=2-1]
			\end{tikzcd}
		\end{equation}
		Following the top path, one gets instead:
		\begin{equation}
			S \times \Sigma \nto{\delta} S
			\quad\mapsto\quad
			\begin{tikzcd}[ampersand replacement=\&]
				{\{s \nreachto{w}_{\delta} s', w \in \Sigma^*\}} \\
				{\Sigma^*}
				\arrow["{\pi_{\Sigma^*}}", from=1-1, to=2-1]
			\end{tikzcd}
			\quad\mapsto\quad
			\begin{tikzcd}[ampersand replacement=\&, column sep=small]
				{\{s \nreachto{w}_{\delta} s', w = p^*(v), v \in \Xi^*\}} \& {\{s \nreachto{w}_{\delta} s', w \in \Sigma^*\}} \\
				{\Xi^*} \& {\Sigma^*}
				\arrow["{\pi_{\Sigma^*}}", from=1-2, to=2-2]
				\arrow["{p^*}"', from=2-1, to=2-2]
				\arrow["{\Set/p*(\pi_{\Sigma^*})}"', from=1-1, to=2-1]
				\arrow[from=1-1, to=1-2]
				\arrow["\lrcorner"{anchor=center, pos=0.125, rotate=45}, draw=none, from=1-1, to=2-2]
			\end{tikzcd}
		\end{equation}
		Thus the components of $P_p$ are:
		\begin{equation}
			% https://q.uiver.app/?q=WzAsMyxbMSwxLCJcXFhpXioiXSxbMiwwLCJcXHtzIFxcbnJlYWNodG97d31fe1xcZGVsdGF9IHMnLCB3ID0gcF4qKHYpLCB2IFxcaW4gXFxYaV4qXFx9Il0sWzAsMCwiXFx7cyBcXG5yZWFjaHRve3Z9X3tcXFRyYW5zKHApKFxcZGVsdGEpfSBzJywgdiBcXGluIFxcWGleKlxcfSJdLFsxLDBdLFsyLDBdLFsyLDEsIlJfcChcXGRlbHRhKSJdXQ==
			\begin{tikzcd}[ampersand replacement=\&, sep=scriptsize]
				{\{s \nreachto{v}_{\Trans(p)(\delta)} s', v \in \Xi^*\}} \&\& {\{s \nreachto{w}_{\delta} s', w = p^*(v), v \in \Xi^*\}} \\
				\& {\Xi^*}
				\arrow[from=1-3, to=2-2]
				\arrow[from=1-1, to=2-2]
				\arrow["{P_p(\delta)}", from=1-1, to=1-3]
			\end{tikzcd}
		\end{equation}
		simply by setting $P_p(\delta)(s \nreachto{v}_{\Trans(p)(\delta)} s') = s \nreachto{p(v)}_{\delta} s'$. Clearly this assignment is natural in $\delta$ (since maps of labelled transition systems commute with paths) and $P_p \comp P_q = P_{p \comp q}$ (vertical functoriality).
		\item The naturality squares for $P$ are given, for each horizontal $h:\Sigma \to \Sigma'$ in $\Alph$, by
		\begin{equation}
			% https://q.uiver.app/?q=WzAsNCxbMCwwLCJcXFRyYW5zKFxcU2lnbWEpIl0sWzAsMSwiXFxUcmFucyhcXFNpZ21hJykiXSxbMSwwLCJcXFNldC9cXFNpZ21hXioiXSxbMSwxLCJcXFNldC97XFxTaWdtYSd9XioiXSxbMCwxLCJcXFRyYW5zKGgpIiwyLHsic3R5bGUiOnsiYm9keSI6eyJuYW1lIjoiYmFycmVkIn19fV0sWzIsMywiXFxTZXQvaCIsMCx7InN0eWxlIjp7ImJvZHkiOnsibmFtZSI6ImJhcnJlZCJ9fX1dLFswLDIsIlJfXFxTaWdtYSJdLFsxLDMsIlJfe1xcU2lnbWEnfSIsMl0sWzQsNSwiUl9oIiwxLHsic2hvcnRlbiI6eyJzb3VyY2UiOjIwLCJ0YXJnZXQiOjIwfX1dXQ==
			\begin{tikzcd}[ampersand replacement=\&]
				{\Trans(\Sigma)} \& {\Set/\Sigma^*} \\
				{\Trans(\Sigma')} \& {\Set/{\Sigma'}^*}
				\arrow[""{name=0, anchor=center, inner sep=0}, "{\Trans(h)}"', "\shortmid"{marking}, from=1-1, to=2-1]
				\arrow[""{name=1, anchor=center, inner sep=0}, "{\Set/h^*}", "\shortmid"{marking}, from=1-2, to=2-2]
				\arrow["{P_\Sigma}", from=1-1, to=1-2]
				\arrow["{P_{\Sigma'}}"', from=2-1, to=2-2]
				\arrow["{P_h}"{description}, shorten <=10pt, shorten >=10pt, Rightarrow, from=0, to=1]
			\end{tikzcd}
		\end{equation}
		defined as
		\begin{eqalign}
			P_\alpha(\delta, \zeta) : \Trans(h)(\delta, \zeta) &\longto \Set/k^*(P_\Sigma(\delta), P_{\Sigma'}(\zeta))\\
			% https://q.uiver.app/?q=WzAsNCxbMCwwLCJTIFxcdGltZXMgXFxTaWdtYSAiXSxbMCwxLCJTIl0sWzEsMCwiVCBcXHRpbWVzIFxcU2lnbWEnIl0sWzEsMSwiVCJdLFsxLDMsIlxcdmFycGhpIl0sWzAsMiwiXFx2YXJwaGkgXFx0aW1lcyBoIl0sWzAsMSwiXFxkZWx0YSIsMl0sWzIsMywiXFx6ZXRhIl1d
			\begin{tikzcd}[ampersand replacement=\&, sep=scriptsize]
				{S \times \Sigma } \& {T \times \Sigma'} \\
				S \& T
				\arrow["\varphi", from=2-1, to=2-2]
				\arrow["{\varphi \times h}", from=1-1, to=1-2]
				\arrow["\delta"', from=1-1, to=2-1]
				\arrow["\zeta", from=1-2, to=2-2]
			\end{tikzcd}
			&\longmapsto
			\begin{tikzcd}[ampersand replacement=\&, sep=scriptsize]
				{\{s \nreachto{v}_\delta s'\}} \& {\{s \nreachto{w}_\zeta s'\}} \\
				{\Sigma^*} \& {{\Sigma'}^*}
				\arrow["{h^*}"', from=2-1, to=2-2]
				\arrow["{\pi_{{\Sigma'}^*}}", from=1-2, to=2-2]
				\arrow["{\pi_{\Sigma^*}}"', from=1-1, to=2-1]
				\arrow["{P_h(\varphi)}", from=1-1, to=1-2]
			\end{tikzcd}
		\end{eqalign}
		where $P_h(s \nreachto{v}_\delta s') = \varphi(s) \nreachto{h^*(v)}_\zeta \varphi(s')$. This is well-defined since, by definition, $\varphi$ preserve transitions. We leave the reader to prove horizontal functoriality.
	\end{enumerate}
\end{example}

\begin{remark}
\label{rmk:states-to-reach}
	We can already see an example of \emph{map} of behaviours. In fact consider the behaviour of states $\states : \Trans \to \BehSys$ and the behaviour of paths $\systh P : \Trans \to \BehSys$:
	\begin{equation}
		% https://q.uiver.app/?q=WzAsNSxbMCwwLCJcXEFscGheXFx0b3AiXSxbMCwyLCJcXFNwYW4oXFxTZXQpXlxcdG9wIl0sWzIsMSwiXFxkYmxDYXQiXSxbMSwwXSxbMSwyXSxbMCwxLCIxXlxcdG9wIiwyLHsiY3VydmUiOjN9XSxbMCwyLCJcXFRyYW5zIiwwLHsiY3VydmUiOi0xfV0sWzEsMiwiXFxTZXQvLSIsMix7ImN1cnZlIjoxfV0sWzMsNCwiUiIsMCx7ImxhYmVsX3Bvc2l0aW9uIjo0MCwib2Zmc2V0IjotNCwiY3VydmUiOi0yLCJzaG9ydGVuIjp7InNvdXJjZSI6NDAsInRhcmdldCI6NDB9LCJsZXZlbCI6Mn1dLFswLDEsInsoLSleKn1eXFx0b3AiLDAseyJjdXJ2ZSI6LTN9XSxbMyw0LCJcXHN0YXRlcyIsMix7ImxhYmVsX3Bvc2l0aW9uIjozMCwib2Zmc2V0Ijo0LCJjdXJ2ZSI6Miwic2hvcnRlbiI6eyJzb3VyY2UiOjQwLCJ0YXJnZXQiOjQwfSwibGV2ZWwiOjJ9XSxbNSw5LCJcXHZhcmVwc2lsb24iLDIseyJzaG9ydGVuIjp7InNvdXJjZSI6MjAsInRhcmdldCI6MjB9fV0sWzEwLDgsIlxcRXBzaWxvbiIsMCx7InNob3J0ZW4iOnsic291cmNlIjoyMCwidGFyZ2V0IjoyMH19XV0=
		\begin{tikzcd}[ampersand replacement=\&]
			{\Alph^\top} \& {} \\
			\&\& \dblCat \\
			{\Span(\Set)^\top} \& {}
			\arrow[""{name=0, anchor=center, inner sep=0}, "{1^\top}"', curve={height=18pt}, from=1-1, to=3-1]
			\arrow["\Trans", curve={height=-15pt}, from=1-1, to=2-3]
			\arrow["{\Set/-}"', curve={height=15pt}, from=3-1, to=2-3]
			\arrow[""{name=1, anchor=center, inner sep=0}, "{\ P}"{pos=0.42}, shift left=4, curve={height=-12pt}, shorten <=25pt, shorten >=25pt, Rightarrow, from=1-2, to=3-2]
			\arrow[""{name=2, anchor=center, inner sep=0}, "{{(-)^*}^\top}", curve={height=-18pt}, from=1-1, to=3-1]
			\arrow[""{name=3, anchor=center, inner sep=0}, "{\states\ }"'{pos=0.35}, shift right=4, curve={height=12pt}, shorten <=18pt, shorten >=18pt, Rightarrow, from=1-2, to=3-2]
			\arrow["\varepsilon^\top"', shorten <=7pt, shorten >=7pt, Rightarrow, from=0, to=2]
			\arrow["\varepsilon^\flat", shorten <=8pt, shorten >=8pt, Rightarrow, from=3, to=1]
		\end{tikzcd}
	\end{equation}
	The horizontal transformation $\varepsilon : 1 \twoto (-)^*$ picks out the empty word on each set of strings. The modification $\varepsilon^\flat$ is a square in $\dblCat$:
	\begin{equation}
		% https://q.uiver.app/?q=WzAsNCxbMCwwLCJcXFRyYW5zKFxcU2lnbWEpIl0sWzEsMCwiXFxTZXQvMSJdLFswLDEsIlxcVHJhbnMoXFxTaWdtYSkiXSxbMSwxLCJcXFNldC9cXFNpZ21hXioiXSxbMCwxLCJcXHN0YXRlc19cXFNpZ21hIl0sWzIsMywiXFxzdGF0ZXNfXFxzaWdtYSIsMl0sWzEsMywiXFxTZXQvXFx2YXJlcHNpbG9uX1xcU2lnbWEiLDAseyJzdHlsZSI6eyJib2R5Ijp7Im5hbWUiOiJiYXJyZWQifX19XSxbMCwyLCIiLDIseyJsZXZlbCI6Miwic3R5bGUiOnsiYm9keSI6eyJuYW1lIjoiYmFycmVkIn0sImhlYWQiOnsibmFtZSI6Im5vbmUifX19XSxbNyw2LCJcXHZhcmVwc2lsb25eXFxmbGF0X1xcU2lnbWEiLDAseyJzaG9ydGVuIjp7InNvdXJjZSI6MjAsInRhcmdldCI6MjB9fV1d
		\begin{tikzcd}[ampersand replacement=\&]
			{\Trans(\Sigma)} \& {\Set/1} \\
			{\Trans(\Sigma)} \& {\Set/\Sigma^*}
			\arrow["{\states_\Sigma}", from=1-1, to=1-2]
			\arrow["{P_\Sigma}"', from=2-1, to=2-2]
			\arrow[""{name=0, anchor=center, inner sep=0}, "{\Set/\varepsilon_\Sigma}", "\shortmid"{marking}, from=1-2, to=2-2]
			\arrow[""{name=1, anchor=center, inner sep=0}, "\shortmid"{marking}, Rightarrow, no head, from=1-1, to=2-1]
			\arrow["{\varepsilon^\flat_\Sigma}", shorten <=10pt, shorten >=10pt, Rightarrow, from=1, to=0]
		\end{tikzcd}
	\end{equation}
	where
	\begin{eqalign}
		\varepsilon^\flat_\Sigma(\delta,\zeta) : \Trans(\Sigma)(\delta,\zeta) &\longto \Set/\varepsilon_\Sigma(\states_\Sigma(\delta), P_\Sigma(\zeta))\\
		% https://q.uiver.app/?q=WzAsNCxbMCwwLCJTIFxcdGltZXMgXFxTaWdtYSJdLFswLDEsIlMiXSxbMSwwLCJUIFxcdGltZXMgXFxTaWdtYSJdLFsxLDEsIlQiXSxbMCwxLCJcXGRlbHRhIiwyXSxbMiwzLCJcXHpldGEiXSxbMSwzLCJcXHZhcnBoaSIsMl0sWzAsMiwiXFx2YXJwaGkgXFx0aW1lcyBcXFNpZ21hIl1d
		\begin{tikzcd}[ampersand replacement=\&,sep=scriptsize]
			{S \times \Sigma} \& {T \times \Sigma} \\
			S \& T
			\arrow["\delta"', from=1-1, to=2-1]
			\arrow["\zeta", from=1-2, to=2-2]
			\arrow["\varphi"', from=2-1, to=2-2]
			\arrow["{\varphi \times \Sigma}", from=1-1, to=1-2]
		\end{tikzcd}
		&\longmapsto
		% https://q.uiver.app/?q=WzAsNCxbMCwwLCJcXHN0YXRlc19cXFNpZ21hKFxcZGVsdGEpIl0sWzEsMCwiUl9cXFNpZ21hKFxcZGVsdGEsXFx6ZXRhKSJdLFswLDEsIjEiXSxbMSwxLCJcXFNpZ21hXioiXSxbMiwzLCJcXHZhcmVwc2lsb24iXSxbMSwzLCJcXHBpX3tcXFNpZ21hXip9Il0sWzAsMiwiISIsMl0sWzAsMSwiXFx2YXJlcHNpbG9uXlxcZmxhdF97XFxkZWx0YSxcXHpldGF9Il1d
		\begin{tikzcd}[ampersand replacement=\&,sep=scriptsize]
			{\states_\Sigma(\delta)} \& {P_\Sigma(\delta,\zeta)} \\
			1 \& {\Sigma^*}
			\arrow["\varepsilon", from=2-1, to=2-2]
			\arrow["{\pi_{\Sigma^*}}", from=1-2, to=2-2]
			\arrow["{!}"', from=1-1, to=2-1]
			\arrow["{\varepsilon^\flat_{\delta,\zeta}}", from=1-1, to=1-2]
		\end{tikzcd}
	\end{eqalign}
	is given by $\varepsilon^\flat_{\delta,\zeta}(s) = (\varepsilon, \varphi(s), \varphi(s))$.
\end{remark}

\begin{example}[Fixpoints]
	\matteo{TODO}
\end{example}

\begin{example}[Trajectories]
	\matteo{TODO} % integral curves of vector fields, traces of open dynamical systems
\end{example}

\begin{example}[Languages]
	One of the first notions of `behaviour of a system' one encounters in computer science is that of \emph{language of a finite states machines}. These, indeed, give a theory of behaviours for the theory of finite states machine we described in~\cref{ex:fsm}.

	Given an FSM $\sys S = (S, \Sigma, \delta, S^*)$, its language is the subset of $\Sigma^*$ given by those words which, when fed to $\sys S$, leave it in an accepting state (one in $S^*$). More precisely, we first have to pick an initial state $s_0 \in S$, then we can start using $\delta$, inputting the given word one character at a time, until we are done. Thus the language of an $\sys S$ is really a family of subsets of $\Sigma^*$, given by the recursively-defined map
	\begin{eqalign}
		L_{\sys S} : S &\longto \pow\Sigma^*\\
		s_0 &\longmapsto \{\sigma_1\cdots\sigma_n \suchthat \sigma_2\cdots\sigma_n \in L_{\sys S}(\delta(s_0, \sigma_1)) \}.
	\end{eqalign}
	Notice this family of subsets can also be defined as a `bundle of sets' (a map we use for its fibers)
	\begin{equation}
		\Lambda(\sys S) := \{ (s_0, w) \mid \in w \in L_{\sys S}(s_0) \} \monoto S \times \Sigma^* \nlongto{\pi_{\Sigma^*}} \Sigma^* \in \Set/\Sigma^*.
	\end{equation}
	Thus we see there is a theory of behaviour
	\begin{equation}
		% https://q.uiver.app/?q=WzAsNSxbMCwwLCJcXEFscGheXFx0b3AiXSxbMCwyLCJcXFNwYW4oXFxTZXQpXlxcdG9wIl0sWzIsMSwiXFxkYmxDYXQiXSxbMSwwXSxbMSwyXSxbMCwxLCJ7KC0pXip9XlxcdG9wIiwyXSxbMCwyLCJcXEZTTSIsMCx7ImN1cnZlIjotMX1dLFsxLDIsIlxcU2V0Ly0iLDIseyJjdXJ2ZSI6MX1dLFszLDQsIkwiLDAseyJvZmZzZXQiOjUsInNob3J0ZW4iOnsic291cmNlIjozMCwidGFyZ2V0IjozMH0sImxldmVsIjoyfV1d
		\begin{tikzcd}[ampersand replacement=\&, sep=small]
			{\Alph^\top} \& {} \\
			\&\& \dblCat \\
			{\Span(\Set)^\top} \& {}
			\arrow["{{(-)^*}^\top}"', from=1-1, to=3-1]
			\arrow["\FSM", curve={height=-6pt}, from=1-1, to=2-3]
			\arrow["{\Set/-}"', curve={height=6pt}, from=3-1, to=2-3]
			\arrow["\Lambda", shift right=5, shorten <=10pt, shorten >=15pt, Rightarrow, from=1-2, to=3-2]
		\end{tikzcd}
	\end{equation}
	where
	\begin{enumerate}
		\item $(-)^* : \Alph \to \Span(\Set)$ is the same as in~\cref{ex:paths},
		\item the component at $\Sigma : \Alph$ of $\Lambda$ is given by
		\begin{eqalign}
		\label{eq:fsm-lang}
			\Lambda_\Sigma : \FSM(\Sigma) &\longto \Set/\Sigma^*\\
			(S, \Sigma, \delta, S^*) &\longmapsto \Lambda(\sys S) : \{ (s_0, w) \mid w \in L_{\sys S}(s_0) \} \to \Sigma^*
		\end{eqalign}
	\end{enumerate}

	The functoriality properties of these assignments can be proven by making the following observation. For each $n \in \N$ (including $0$), there are finite states machines $\sys{Fin\, n} = (\Fin\, n+1, \Fin\, n, \_+1, \{n\})$. Explicitly, these have $n+1$ states and are defined over an alphabet with $n$ symbols. The transition map is
	\begin{eqalign}
		\_ + 1 : \Fin\,n+1\ \times\ \Fin\, n &\longto \Fin\,n+1\\
		(k, i) &\longmapsto k+1
	\end{eqalign}
	and the only accepting state is $n \in \Fin\,n+1$.

	A map from this machine to another like $\sys S$, mediated by a map of alphabets $\sigma : \Fin\,n \to \Sigma$, looks like this:
	\begin{equation}
		% https://q.uiver.app/?q=WzAsNixbMCwwLCJcXEZpblxcLCBuKzEgXFx0aW1lcyBcXEZpblxcLG4iXSxbMCwyLCJcXEZpblxcLG4rMSJdLFsyLDAsIlMgXFx0aW1lcyBcXFNpZ21hIl0sWzIsMiwiUyJdLFsyLDMsIlNeKiJdLFswLDMsIlxce25cXH0iXSxbMiwzLCJcXGRlbHRhIiwyXSxbMCwxLCJcXF8rMSIsMl0sWzAsMiwicyJdLFsxLDMsIlxcc2lnbWEiXSxbNSw0LCJcXHN1YnNldGVxIiwzLHsic3R5bGUiOnsiYm9keSI6eyJuYW1lIjoibm9uZSJ9LCJoZWFkIjp7Im5hbWUiOiJub25lIn19fV0sWzUsMSwiXFxzdWJzZXRlcSIsMyx7InN0eWxlIjp7ImJvZHkiOnsibmFtZSI6Im5vbmUifSwiaGVhZCI6eyJuYW1lIjoibm9uZSJ9fX1dLFs0LDMsIlxcc3Vic2V0ZXEiLDMseyJzdHlsZSI6eyJib2R5Ijp7Im5hbWUiOiJub25lIn0sImhlYWQiOnsibmFtZSI6Im5vbmUifX19XV0=
		\begin{tikzcd}[ampersand replacement=\&, sep=small]
			{\Fin\, n+1 \times \Fin\,n} \&\& {S \times \Sigma} \\
			\\
			{\Fin\,n+1} \&\& S \\
			{\{n\}} \&\& {S^*}
			\arrow["\delta"', from=1-3, to=3-3]
			\arrow["{\_+1}"', from=1-1, to=3-1]
			\arrow["s \times \sigma", from=1-1, to=1-3]
			\arrow["s", from=3-1, to=3-3]
			\arrow["\subseteq"{marking}, draw=none, from=4-1, to=4-3]
			\arrow["\subseteq"{marking}, draw=none, from=4-1, to=3-1]
			\arrow["\subseteq"{marking}, draw=none, from=4-3, to=3-3]
		\end{tikzcd}
	\end{equation}
	Concretely, this is a map $s: \Fin\,n+1 \to S$ selecting a sequence of $n+1$ states $s_0, \ldots, s_n$ such that
	\begin{equation}
		\forall 0 \leq k \leq n-1,\ s_{k+1} = \delta(s_k, \sigma_k) \quad\text{and}\quad s_n \in S^*.
	\end{equation}
	In other words, this data amounts to a word $\sigma_1 \cdots \sigma_n \in \Sigma^*$ and a state $s_0 \in S$ such that $\sys S$ accepts $\sigma_1 \cdots \sigma_n$ in $n$ steps when starting from $s_0$.

	Notice this also works for the empty word: a map from $\sys {Fin\, 0}$ is given by just a choice of state $s_0 \in S^*$, meaning $\sys S$ accepts the empty word when starting from $s_0$.

	Therefore we see $((-)^*, \Lambda)$ is a sum of corepresentable behaviours, namely $\sum_{n \geq 0} \FSM(\sys {Fin\, n}, -)$. Observe the symmetry restored: $(-)^*$ is also the functor $\sum_{n \geq 0} \Set(\Fin\, n, -)$!
\end{example}

\begin{remark}
	Notice the observation that~\eqref{eq:fsm-lang} is a sum of corepresentables was already valid for the behaviour of paths $\systh P$ in~\cref{ex:paths}.
	The systems corepresenting the behaviours are the same as the ones just described except for the acceptance predicate, which is not there for labelled transition systems.

	In this light, the map described in~\cref{rmk:states-to-reach} is simply the inclusion of $\states = \Trans(\sys{Fin\, 0}, -)$ into the sum $\systh P = \sum_{n \geq 0} \Trans(\sys {Fin\, n}, -)$.
\end{remark}

\subsection{Generalized behaviours}
Generalized behaviours are those landing in a generalized behavioural theory (\cref{ex:gen-behav-systems}).

\begin{example}[Reachability]
\label{ex:reach}
	In~\cref{ex:paths} we've seen finite-length paths are a theory of behaviour for labelled transition systems. If one were to care just about reachability between states though, paths would be overkill: they remember, yes, whether a certain state is reachable from another one, but also all possible way in which this can happen. This can be a plus in some settings, but a burden in others.

	To forget about paths and only keep the reachability relation we have to truncate the information we have about the type of paths between two states to just a proposition.
	In other words, we want to turn our dependent types to preidcates. Doing so will yield a non-standard behaviour of reachability.

	Truncation can be expressed as a morphism of system theories $\|-\|_0 : \BehSys \to \BehSys_0$, where the latter is a theory like $\BehSys$ except on objects, where $\Set/-$ is replaced by $\Sub(-)$. Hence systems are not display maps anymore, but just subobjects.

	The natural transformation $\im : \Set/- \to \Sub(-)$, which sends a display map $\pi:S \to A$ to its image $\im \pi \subseteq A$, induces the desired $\|-\|_0$.

	Now reachability is the non-standard behaviour $\systh R: \Trans \to \BehSys_0$ obtained as $\|\systh P\|_0$, where $\systh P$ is the behaviur of paths from~\cref{ex:paths}.
\end{example}

\subsection{Doctrines of behaviour}
Like theories of systems, theories of behaviours can be gathered in uniform \emph{doctrines} which functorially specify a way to look at theories of a doctrine of systems:

\begin{definition}
	A \textbf{doctrine of behaviour} for the doctrine $\dblcat{Doctrine}$ is map of doctrines of systems into the behavioural doctrine $\BehSys$:
	\begin{equation}
		% https://q.uiver.app/?q=WzAsNSxbMCwwLCJcXGRibGNhdHtEYXRhfSJdLFswLDIsIlxcZGJsY2F0e0NhcnRDYXR9Il0sWzIsMSwiXFxUaGVvcmllcyJdLFsxLDBdLFsxLDJdLFswLDEsIlxcZGJsY2F0IEJeXFx0b3AiLDJdLFswLDIsIlxcZGJsY2F0e0RvY3RyaW5lfSIsMCx7ImN1cnZlIjotMX1dLFsxLDIsIlxcT2JzIiwyLHsiY3VydmUiOjF9XSxbMyw0LCJcXGRibGNhdCBCXlxcZmxhdCIsMCx7Im9mZnNldCI6NSwic2hvcnRlbiI6eyJzb3VyY2UiOjMwLCJ0YXJnZXQiOjMwfSwibGV2ZWwiOjJ9XV0=
		\begin{tikzcd}[ampersand replacement=\&, sep=small]
			{\dblcat{Data}} \& {} \\
			\&\& \Theories \\
			{\dblcat{CartCat}} \& {}
			\arrow["{\doc B^\top}"', from=1-1, to=3-1]
			\arrow["{\doc{Doctrine}}", curve={height=-6pt}, from=1-1, to=2-3]
			\arrow["\doc{Beh}"', curve={height=6pt}, from=3-1, to=2-3]
			\arrow["{\doc B^\flat}"', shift right=3, shorten <=10pt, shorten >=14pt, Rightarrow, from=1-2, to=3-2]
		\end{tikzcd}
	\end{equation}
\end{definition}

\begin{remark}
	The laxity of $B^\flat$ relates the behaviours of the parts of a system to the behaviour of the whole system. The non-invertibility of such a map witnesses emergent behaviours.
	In~\cite[Theorem 5.3.3.1]{myers_categorical_2022}, Myers proves that a large class of behaviours for the doctrine of Moore machines does not, in fact, exhibit emergence, by showing such the laxity of $B^\flat$ is invertible.
\end{remark}

\begin{example}
\label{ex:corep-beh}
	A large class of behaviours are corepresentables, i.e.~defined by simulations of an archetypal system exhibiting that behaviour.
	Thus there is a \textbf{doctrine of corepresentable behaviour} on the doctrine of pointed theories:
	\begin{equation}
		% https://q.uiver.app/?q=WzAsNSxbMCwwLCJcXFRoZW9yaWVzX1xcYXN0Il0sWzAsMiwiXFxkYmxjYXR7Q2FydENhdH0iXSxbMiwxLCJcXFRoZW9yaWVzIl0sWzEsMF0sWzEsMl0sWzAsMSwiXFxTZXQiLDJdLFswLDIsIlUiLDAseyJjdXJ2ZSI6LTF9XSxbMSwyLCJcXE9icyIsMix7ImN1cnZlIjoxfV0sWzMsNCwiXFxIb21eaCIsMix7Im9mZnNldCI6Mywic2hvcnRlbiI6eyJzb3VyY2UiOjMwLCJ0YXJnZXQiOjMwfSwibGV2ZWwiOjJ9XV0=
		\begin{tikzcd}[ampersand replacement=\&, sep=small]
			{\Theories_\ast} \& {} \\
			\&\& \Theories \\
			{\dblcat{CartCat}} \& {}
			\arrow["\Set"', from=1-1, to=3-1]
			\arrow["\doc U", curve={height=-6pt}, from=1-1, to=2-3]
			\arrow["\doc{Beh}"', curve={height=6pt}, from=3-1, to=2-3]
			\arrow["{\Hom^h}"'{pos=0.45}, shift right=3, shorten <=10pt, shorten >=14pt, Rightarrow, from=1-2, to=3-2]
		\end{tikzcd}
	\end{equation}
	An object of $\Theories_\ast$ is a systems theory equipped with a distinguished system (hence a pair of an interface and a system over it) which is used as archetype for a certain kind of behaviour.

	The transformation $\Hom^h$ at a pointed theory $(\Sys : \Comp^\top \to \dblCat, \sys B : \Sys(I))$ is the map of system theories $\Comp^h(\sys B, -) : \Sys \to \BehSys(\Set)$ defined as follows.
	Its two components are the horizontal hom-functor $\Comp^h(I, -) : \Comp \to \dblSet$ and the similar fiberwise hom-functor
	\begin{equation}
		\Comp^h(\sys B, -)^\flat : \Sys \twoto \Set/\Comp^h(I, -).
	\end{equation}
	This latter functor sends a system $\sys S : \Sys(J)$ to the $\Comp^h(I, J)$-indexed family of sets sending a map of interfaces $k : I \to J$ to the set $\Sys(k)(\sys B, \sys S)$ of maps of systems mediated by $k$.
\end{example}

\begin{example}
	Let $\dblcat{MooreData}^!$ be the 2-category of fibrations of categories with a terminal object and a section thereof. These amount to a fibration $p:\cat E \to \cat C$ such that $p(1_{\cat E}) = 1_{\cat C}$, and a section $T : \cat C \to \cat E$ such that $T(1_{\cat C}) = 1_{\cat E}$.
	In this situation, one can build the Moore machine $\fix : \lens{T1}{1} \equalto \lens{1}{1}$ which `does nothing'. Hence we get a \emph{doctrine of fixpoints} by using such a machine as the archetype for the behaviour of a still system:
	\begin{equation}
		% https://q.uiver.app/?q=WzAsNixbMCwxLCJcXFRoZW9yaWVzX1xcYXN0Il0sWzAsMywiXFxkYmxjYXR7Q2FydENhdH0iXSxbMiwyLCJcXFRoZW9yaWVzIl0sWzEsMV0sWzEsM10sWzAsMCwiXFxkYmxjYXR7TW9vcmVEYXRhfV4xIl0sWzAsMSwiXFxTZXQiLDJdLFswLDIsIlUiLDAseyJjdXJ2ZSI6LTF9XSxbMSwyLCJcXE9icyIsMix7ImN1cnZlIjoxfV0sWzMsNCwiXFxIb21eaCIsMix7Im9mZnNldCI6Mywic2hvcnRlbiI6eyJzb3VyY2UiOjMwLCJ0YXJnZXQiOjMwfSwibGV2ZWwiOjJ9XSxbNSwyLCJcXE1vb3JlIiwwLHsiY3VydmUiOi0zfV0sWzUsMCwiKFxcTW9vcmUsIFxcZml4IDogXFxNb29yZSgxKSkiLDJdXQ==
		\begin{tikzcd}[ampersand replacement=\&, sep=small]
			{\dblcat{MooreData}^!} \\[2ex]
			{\Theories_\ast} \& {} \\
			\&\& \Theories \\
			{\dblcat{CartCat}} \& {}
			\arrow["\Set"', from=2-1, to=4-1]
			\arrow["\doc U", curve={height=-6pt}, from=2-1, to=3-3]
			\arrow["\doc{Beh}"', curve={height=6pt}, from=4-1, to=3-3]
			\arrow["{\Hom^h}"', shift right=3, shorten <=10pt, shorten >=14pt, Rightarrow, from=2-2, to=4-2]
			\arrow["\doc{Moore}", curve={height=-18pt}, from=1-1, to=3-3]
			\arrow["{(\Moore,\ \fix : \Moore(1))}"', from=1-1, to=2-1]
		\end{tikzcd}
	\end{equation}
\end{example}

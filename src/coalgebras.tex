\section{Relation with coalgebraic systems theory}
In the last decades coalgebra has been hailed s a universal theory of systems~\cite{rutten_universal_2000}.
In this section we remark on how the present framework integrates the coalgebraic one.

Let us start by recalling the setup of coalgebrac systems theory.
There, systems are coalgebras of an endofunctor $F:\cat S \to \cat S$ where $\cat S$ is often $\Set$ but can be taken to be any category of choice.
A coalgebra $X \nto{\delta} FX$ is an `effectful dynamical system', with $F$ playing both the role of interface, (e.g.~Moore machines with interface $\lens{I}{O}$ are coalgebras of $O \times (-)^I$) and of effect theory (e.g.~non-deterministic Moore machines with interface $\lens{I}{O}$ are coalgebras of $O \times P(-)^I$).

Coalgebraic systems theory takes then mainly place in the category $\Coalg(F)$, or related ones.
Thus systems are routinely compared by morphisms, and this makes the theory particularly structural.
However, composition of systems is completely absent (but this doesn't mean compositionality is ignored, see e.g.~\cite{turi_towards_1997}).

\matteo{...}

\begin{example}
\label{ex:coalgebras}
	We can think of a coalgebra $A \to FA$ as a system with states $A$ and interface $F$. Natural transformations $\alpha : F \Rightarrow F'$ are `lenses' and one gets an indexed
	category
	\begin{equation}
		\Coalg : \End(\cat C) \to \Cat
	\end{equation}
	If $\cat C$ is additionally finitely complete, we can go further and add another dimension. In fact, in this case, $\End(\cat C)$ is fibred over $\cat C$ by evaluation at the terminal object:
	\begin{equation}
		-(1) : \End(\cat C) \to \cat C
	\end{equation}
	The cartesian lift of a given arrow $f:A \to G(1)$ is given by a
	natural transformation $f_G : f^*G \Rightarrow G$ obtained from the
	pullback square
	\begin{equation}
		% https://q.uiver.app/?q=WzAsNCxbMSwwLCJHWCJdLFsxLDEsIkcxIl0sWzAsMSwiQSJdLFswLDAsImZeKkdYIl0sWzAsMSwiRyEiXSxbMiwxLCJmIiwyXSxbMywyXSxbMywwLCJmX3tHLFh9Il0sWzMsMSwiIiwxLHsic3R5bGUiOnsibmFtZSI6ImNvcm5lciJ9fV1d
		\begin{tikzcd}[ampersand replacement=\&, sep=scriptsize]
			{f^*GX} \& GX \\
			A \& G1
			\arrow["{G!}", from=1-2, to=2-2]
			\arrow["f"', from=2-1, to=2-2]
			\arrow[from=1-1, to=2-1]
			\arrow["{f_{G,X}}", from=1-1, to=1-2]
			\arrow["\lrcorner"{anchor=center, pos=0.125}, draw=none, from=1-1, to=2-2]
		\end{tikzcd}
	\end{equation}
	that simultaneously defines $f^*G$ (on morphisms is defined by
	pullback again) and $f_G$.

	The fibred subcategory of polynomial functors is what gives
	`dependent' lenses, whose opposite is the codomain fibration,
	i.e.~`dependent' charts~\cite{spivak_generalized_2019}.
	This suggests that taking the opposite fibration of $-(1)$ gives us a
	fibration of `generalized charts'.

	We can explicitly construct these things if we work out the cartesian
	factorization system induced by $-(1)$ on $\End(\cat C)$. This is
	given by

	\begin{enumerate}
		\item
			Cartesian maps are given by natural transformations whose naturality
			is witness by pullback squares, as suggested by the definition of
			$f^*G$ above: which we make explicit here:

			\begin{equation}
				% https://q.uiver.app/?q=WzAsNCxbMCwwLCJGWCJdLFsxLDAsIkdYIl0sWzEsMSwiR1kiXSxbMCwxLCJGWSJdLFszLDIsIlxcYWxwaGFfWSIsMl0sWzEsMiwiR2YiXSxbMCwxLCJcXGFscGhhX1giXSxbMCwzLCJGZiIsMl0sWzAsMiwiIiwxLHsic3R5bGUiOnsibmFtZSI6ImNvcm5lciJ9fV1d
				\begin{tikzcd}[ampersand replacement=\&, sep=scriptsize]
					FX \& GX \\
					FY \& GY
					\arrow["{\alpha_Y}"', from=2-1, to=2-2]
					\arrow["Gf", from=1-2, to=2-2]
					\arrow["{\alpha_X}", from=1-1, to=1-2]
					\arrow["Ff"', from=1-1, to=2-1]
					\arrow["\lrcorner"{anchor=center, pos=0.125}, draw=none, from=1-1, to=2-2]
				\end{tikzcd}
			\end{equation}
		\item
			Vertical maps are given by natural transformations whose component at
			$1$ is an isomorphism (think: the identity)
	\end{enumerate}

	We define a \emph{generalized chart} $\lens{k^\flat}{k}: F \chartto G$ to be a span in $\End(\cat C)$ whose left leg is vertical and whose right leg is cartesian. Generalized charts look like this:
	\begin{equation}
		% https://q.uiver.app/?q=WzAsOCxbMSwxLCJGMSJdLFszLDEsIkcxIl0sWzMsMCwiRyJdLFsyLDEsIkYxIl0sWzIsMCwiZl4qRyJdLFsxLDAsIkYiXSxbMCwwLCJcXEVuZChcXGNhdCBDKSJdLFswLDEsIlxcY2F0IEMiXSxbNCwyLCJrX0ciXSxbMywxLCJrIiwyXSxbMCwzLCIiLDIseyJsZXZlbCI6Miwic3R5bGUiOnsiaGVhZCI6eyJuYW1lIjoibm9uZSJ9fX1dLFs0LDUsImteXFxmbGF0IiwyXSxbNiw3LCItKDEpIiwyXV0=
		\begin{tikzcd}[ampersand replacement=\&, sep=scriptsize]
			{\End(\cat C)} \& F \& {k^*G} \& G \\
			{\cat C} \& F1 \& F1 \& G1
			\arrow["{k_G}", from=1-3, to=1-4]
			\arrow["k"', from=2-3, to=2-4]
			\arrow[Rightarrow, no head, from=2-2, to=2-3]
			\arrow["{k^\flat}"', from=1-3, to=1-2]
			\arrow["{-(1)}"', from=1-1, to=2-1]
		\end{tikzcd}
	\end{equation}
	These might look like lenses (because lenses are obtained by opping a
	fibration) but they are actually charts. We can verify this by looking
	at the case in which $F$ and $G$ are polynomial (over $\Set$), to see if this
	construction recovers the usual one. We see that $k^\flat$ lives in
	\begin{equation}
		\Nat\left(\sum_{i \in F1} y^{F_i}, \sum_{j \in F1} y^{G_{k(j)}}\right) \iso \prod_{i \in F1} \sum_{j \in F1} \Set(F_i, G_{k(j)}) \iso \sum_{f : F1 \to F1} \prod_{i \in F1} \Set(F_i, G_{k(f(i))})
	\end{equation}
	and since we know it is a vertical map, then $k^\flat$ actually lives in the subobject of the right hand side for which the map $f$ is the identity. Thus we
	see $\lens{k^\flat}{k}$ encodes the data of a chart (notice how $F$ and
	$G$ swapped places: charts are lenses `relative to lenses').

	% Thus is not far-fetched to think of such a fibration as giving
	% generalized lenses and charts. What is sure is that the double
	% Grothendieck construction of David Jaz Myers gives us a double category
	% of endofunctors, natural transformations (loose), generalized charts
	% (tight) and commutative squares.

	This allows us to extend the indexed category $\Coalg$ defined
	previously to have a profunctorial action. So a given generalized chart
	$\lens{k^\flat}{k} : F \chartto G$ is mapped to a profunctor
	$\Coalg\lens{k^\flat}{k} : \Coalg(F) \profto \Coalg(G)$. This has a rather complex
	definition: it maps two coalgebras $\gamma:A \to FA$ and
	$\delta:B \to GB$ to the set of $\phi:A \to B$ that make the
	following commute:
	\begin{equation}
		% https://q.uiver.app/?q=WzAsOCxbMCwwLCJBIl0sWzEsMiwia14qR0IiXSxbMCwyLCJGQSJdLFsxLDEsImteKkIiXSxbMiwxLCJCIl0sWzEsMywiRjEiXSxbMiwyLCJHQiJdLFsyLDMsIkcxIl0sWzAsMiwiXFxnYW1tYSIsMV0sWzMsMSwiXFxkZWx0YSciLDFdLFszLDQsImsnIiwxXSxbMCw0LCJcXHBoaSIsMSx7ImN1cnZlIjotMiwic3R5bGUiOnsiYm9keSI6eyJuYW1lIjoiZGFzaGVkIn19fV0sWzUsNywiayIsMV0sWzIsNSwiRiEiLDEseyJjdXJ2ZSI6Mn1dLFsxLDUsIkchJyIsMV0sWzQsNiwiXFxkZWx0YSIsMV0sWzMsNiwiIiwwLHsic3R5bGUiOnsibmFtZSI6ImNvcm5lciJ9fV0sWzYsNywiRyEiLDFdLFsxLDYsImtfe0csQn0iLDFdLFsxLDcsIiIsMix7InN0eWxlIjp7Im5hbWUiOiJjb3JuZXIifX1dXQ==
		\begin{tikzcd}[ampersand replacement=\&]
			A \\
			\& {k^*B} \& B \\
			FA \& {k^*GB} \& GB \\
			\& F1 \& G1
			\arrow["\gamma"{description}, from=1-1, to=3-1]
			\arrow["{\delta'}"{description}, from=2-2, to=3-2]
			\arrow["{k'}"{description}, from=2-2, to=2-3]
			\arrow["\phi"{description}, curve={height=-12pt}, dashed, from=1-1, to=2-3]
			\arrow["k"{description}, from=4-2, to=4-3]
			\arrow["{F!}"{description}, curve={height=12pt}, from=3-1, to=4-2]
			\arrow["{G!'}"{description}, from=3-2, to=4-2]
			\arrow["\delta"{description}, from=2-3, to=3-3]
			\arrow["\lrcorner"{anchor=center, pos=0.125}, draw=none, from=2-2, to=3-3]
			\arrow["{G!}"{description}, from=3-3, to=4-3]
			\arrow["{k_{G,B}}"{description}, from=3-2, to=3-3]
			\arrow["\lrcorner"{anchor=center, pos=0.125}, draw=none, from=3-2, to=4-3]
		\end{tikzcd}
	\end{equation}
	\begin{equation}
		% https://q.uiver.app/?q=WzAsNSxbMCwwLCJBIl0sWzMsMSwia14qR0IiXSxbMCwxLCJGQSJdLFszLDAsImteKkIiXSxbMSwxLCJGQiJdLFswLDIsIlxcZ2FtbWEiLDJdLFszLDEsIlxcZGVsdGEnIiwxXSxbMCwzLCJcXGV4aXN0cyEgXFxsYW5nbGUgXFxnYW1tYSBcXGNvbXAgRiEsIFxccGhpIFxccmFuZ2xlIiwxLHsic3R5bGUiOnsiYm9keSI6eyJuYW1lIjoiZG90dGVkIn19fV0sWzEsNCwia15cXGZsYXRfQiIsMix7ImxhYmVsX3Bvc2l0aW9uIjozMH1dLFsyLDQsIkYoXFxwaGkpIiwwLHsic3R5bGUiOnsiYm9keSI6eyJuYW1lIjoiZGFzaGVkIn19fV1d
		\begin{tikzcd}[ampersand replacement=\&]
			A \&\&\& {k^*B} \\
			FA \& FB \&\& {k^*GB}
			\arrow["\gamma"', from=1-1, to=2-1]
			\arrow["{\delta'}"{description}, from=1-4, to=2-4]
			\arrow["{\exists! \langle \gamma \comp F!, \phi \rangle}"{description}, dotted, from=1-1, to=1-4]
			\arrow["{k^\flat_B}"'{pos=0.3}, from=2-4, to=2-2]
			\arrow["{F(\phi)}", dashed, from=2-1, to=2-2]
		\end{tikzcd}
	\end{equation}
	% \begin{equation}
	% 	% https://q.uiver.app/?q=WzAsOSxbMCwwLCJBIl0sWzEsMiwiRkIiXSxbNCwwLCJCIl0sWzQsMiwiR0IiXSxbMywyLCJrXipHQiJdLFswLDIsIkZBIl0sWzMsMSwia14qQiJdLFszLDMsIkYxIl0sWzQsMywiRzEiXSxbMiwzLCJcXGRlbHRhIl0sWzQsMSwia15cXGZsYXRfQiIsMix7ImxhYmVsX3Bvc2l0aW9uIjozMH1dLFs1LDEsIkYoXFxwaGkpIiwwLHsic3R5bGUiOnsiYm9keSI6eyJuYW1lIjoiZGFzaGVkIn19fV0sWzAsNSwiXFxnYW1tYSIsMl0sWzYsMiwiayciXSxbNiw0LCJcXGRlbHRhJyIsMV0sWzYsMywiIiwwLHsic3R5bGUiOnsibmFtZSI6ImNvcm5lciJ9fV0sWzAsMiwiXFxwaGkiLDEseyJzdHlsZSI6eyJib2R5Ijp7Im5hbWUiOiJkYXNoZWQifX19XSxbMCw2LCJcXGV4aXN0cyEgXFxsYW5nbGUgXFxnYW1tYSBcXGNvbXAgRiEsIFxccGhpIFxccmFuZ2xlIiwxLHsic3R5bGUiOnsiYm9keSI6eyJuYW1lIjoiZG90dGVkIn19fV0sWzMsOCwiRyEiXSxbNyw4LCJrIiwyXSxbNSw3LCJGISIsMix7ImN1cnZlIjoyfV0sWzQsNywiRyEnIiwyXSxbNCw4LCIiLDIseyJzdHlsZSI6eyJuYW1lIjoiY29ybmVyIn19XSxbNCwzLCJrX3tHLEJ9Il1d
	% 	\begin{tikzcd}[ampersand replacement=\&]
	% 		A \&\&\&\& B \\
	% 		\&\&\& {k^*B} \\
	% 		FA \& FB \&\& {k^*GB} \& GB \\
	% 		\&\&\& F1 \& G1
	% 		\arrow["\delta", from=1-5, to=3-5]
	% 		\arrow["{k^\flat_B}"'{pos=0.3}, from=3-4, to=3-2]
	% 		\arrow["{F(\phi)}", dashed, from=3-1, to=3-2]
	% 		\arrow["\gamma"', from=1-1, to=3-1]
	% 		\arrow["{k'}", from=2-4, to=1-5]
	% 		\arrow["{\delta'}"{description}, from=2-4, to=3-4]
	% 		\arrow["\lrcorner"{anchor=center, pos=0.125}, draw=none, from=2-4, to=3-5]
	% 		\arrow["\phi"{description}, dashed, from=1-1, to=1-5]
	% 		\arrow["{\exists! \langle \gamma \comp F!, \phi \rangle}"{description}, dotted, from=1-1, to=2-4]
	% 		\arrow["{G!}", from=3-5, to=4-5]
	% 		\arrow["k"', from=4-4, to=4-5]
	% 		\arrow["{F!}"', curve={height=12pt}, from=3-1, to=4-4]
	% 		\arrow["{G!'}"', from=3-4, to=4-4]
	% 		\arrow["\lrcorner"{anchor=center, pos=0.125}, draw=none, from=3-4, to=4-5]
	% 		\arrow["{k_{G,B}}", from=3-4, to=3-5]
	% 	\end{tikzcd}
	% \end{equation}

	Specifically, we are asking for the following:
	\begin{eqalign}
		\forall a \in A,&\quad G!(\delta(\phi(a))) = k(F!(\gamma(a)))\\
		\forall a \in A,&\quad k^\flat_B(\delta'(F!(\gamma(a)), \phi(a))) = F(\phi)(\gamma(a))
	\end{eqalign}

	\matteo{The double category End...}

	All in all, this gives us the \textbf{theory of coalgebras}:
	\begin{equation}
		\Coalg : \dblEnd^\top \unilaxto \dblCat.
	\end{equation}
\end{example}


% \begin{example}[Coalgebraic behaviour]
% 	Let $\cat C$ be a cartesian category. Let $\dblEnd^\nu(\cat C)$ denote the double category of endofunctors described in \cref{ex:coalgebras} but restricted to those endofunctors that admit a final coalgebra.
% 	We thus end up with a restricted theory of coalgebras $\Coalg^\nu : \theoryover{\dblEnd^\nu}$. On this, we can define the following theory of behaviour:
% 	\begin{equation}
% 		% https://q.uiver.app/?q=WzAsNSxbMCwwLCJ7XFxkYmxFbmReXFxudX1eXFx0b3AiXSxbMCwyLCJcXFNwYW4oXFxjYXQgQyleXFx0b3AiXSxbMiwxLCJcXGRibENhdCJdLFsxLDBdLFsxLDJdLFswLDEsIlxcbnUiLDJdLFswLDIsIlxcQ29hbGdeXFxudSIsMCx7ImN1cnZlIjotMX1dLFsxLDIsIlxcY2F0IEMvLSIsMix7ImN1cnZlIjoxfV0sWzMsNCwiISIsMix7ImxhYmVsX3Bvc2l0aW9uIjo0MCwib2Zmc2V0Ijo0LCJzaG9ydGVuIjp7InNvdXJjZSI6MzAsInRhcmdldCI6NTB9LCJsZXZlbCI6Mn1dXQ==
% 		\begin{tikzcd}[ampersand replacement=\&]
% 			{{\dblEnd^\nu(\cat C)}^\top} \& {} \\
% 			\&\& \dblCat \\
% 			{\Span(\cat C)^\top} \& {}
% 			\arrow["\nu"', from=1-1, to=3-1]
% 			\arrow["{\Coalg^\nu}", curve={height=-6pt}, from=1-1, to=2-3]
% 			\arrow["{\cat C/-}"', curve={height=6pt}, from=3-1, to=2-3]
% 			\arrow["{!}"'{pos=0.4}, shift right=4, shorten <=10pt, shorten >=15pt, Rightarrow, from=1-2, to=3-2]
% 		\end{tikzcd}
% 	\end{equation}
% 	This is given by
% 	\begin{enumerate}
% 		\item The lax double functor which sends an endofunctor $F : \dblEnd^\nu$ to the carrier of its final coalgebra $\nu F : \cat C$.

% 		Since $\nu$ is functorial on $\End^\nu(\cat C)$, we can map natural transformations $p:F \twoto G$ (the vertical arrows in $\dblEnd^\nu(\cat C)$) to `functional spans' $\nu F \equalto \nu F \nto{\nu p} \nu G$, and this is again functorial.
% 		In the horizontal direction, we have to prove a generalized chart $\lens{h^\flat}{h} : F \horto G$  also induces a map $\nu \lens{h^\flat}{h} : \nu F \to \nu G$. Remember that such a generalized chart is given by a map $F1 \to G1$ and and by a natural transformation $h^\flat : f^*G \twoto F$ such that $h^\flat_1$ is invertible (without loss of generality, the identity).

% 		Adamek's theorem characterizes the carrier of final coalgebras as limits of the diagrams obtained by iterated application of a functor to the terminal object:
% 		\begin{equation}
% 			\cdots F^2(1) \nto{F!} F1 \nto{!} 1
% 		\end{equation}
% 		Clearly, a map like $h$ induces a morphism between the diagrams corresponding to $F$ and $G$, which in turn induces a morphism $\nu F \to \nu G$.

% 		\item The vertical lax-natural transformation
% 	\end{enumerate}
% \end{example}


\section{Introduction}
% Systems are ubiquitous, in science as in life.
% %People regularly deal with physical systems, political systems, economical systems, living systems, learning systems, writing systems, voting systems, computing systems, etc.
% As we look into a thing, we soon realize it is a system comprised of smaller interacting parts\footnote{Heck, even \emph{atoms}, indivisible par excellence, arise from many layers of interaction among their parts!}. As we walk back, we realize it is itself part of an even more complex system.

% The word `system' roughly means `composite thing' in Greek.
% It seems therefore natural that category theory has something to say about them.

% Given the staggering variety systems come in, it is no surprise scientific paradigms and mathematical models to describe them are manifold and often incompatible with one another.
% Studying each of these paradigms by themselves is surely a useful endeavour, but the ethos of category theory is to adopt a formal, bird-eye view of disciplines, to find the common motifs. Hence in approaching the study of systems theory, the category theorist should ask: what are the formal structures underlying all the different approaches to systems?

Categorical Systems Theory (CST) is a categorical framework for the abstract study of systems irrespective of their contingent aspects.
%It studies any situation involving many parts engaging in some interesting behaviour.
%
Instead of espousing a specific paradigm on the mathematical specification of systems, CST predicates upon the general features such paradigms should have.
In doing so, it captures the essence of the notion of `system'.

The object of study of CST is a \emph{doctrine of systems}, which is defined in \cite{myers_categorical_2022} as way to answer the following questions:
\begin{quote}
	\begin{itemize}
		\item What does it mean to be a system? Does it have a notion of states, or of behaviours? Or is it a diagram describing the way some primitive parts are organized?
		\item What should the interface of a system be?
		\item How can interfaces be connected in composition patterns?
		\item How are systems composed through composition patterns between their interfaces.
		\item What is a map between systems, and how does it affect their interfaces?
		\item When can maps between systems be composed along the same composition patterns as the systems.
	 \end{itemize}
\end{quote}

A doctrine of systems specializes in many different \emph{theories of systems}. For instance there is a doctrine of open dynamical systems encompassing the theory of deterministic dynamical systems, the theory of stochastic dynamical systems, the theory of differential dynamical systems, and many more.
Hence it's usually easier to start describing what a theory of systems is and then to say what does it mean for a doctrine to gather many of them in a single object.

\subsection{References}
Categorical approaches to general systems theory have been around for a long time.
The earliest is probably~\cite{rosen1978fundamentals}, which deals with systems from a behavioural point of view. The work pioneers the idea of gathering systems in categories and considering their observable behaviour as input-output relations.

In the 90s, a series of papers by Katis, Sabadini, Walters and others established a theory of machines \cite{sabadini_functions_1993, bloom_matrices_1996, katis1997bicategories, katis1997span, katis_algebra_1999, katis2002feedback} which anticipated some of the ideas regarding functorial semantics of behaviour, finding categories of matrices are well-adapted to receive such functors.

The works on structured and decorated cospans \cite{fiadeiro2007structured,fong2015decorated, baez2020open,baez2020structured,Baez2022structuredversus} deals with the doctrine of port-plugging systems (circuits) in modern form, and started using some ideas from double category theory to talk about them.
This also happens in \cite{lerman2018networks,culbertson2020formal}.

In \cite{spivak2013operad, libkind2021operadic} and other related works the idea operads provide the syntax for composition of systems is introduced.

Thus we can say that albeit the current form of CST (as the yoga of doubly indexed category and `copresheaves' over them) is due to David Jaz Myers, various pieces of this philosophy have been around for far longer, as well as specific incarnations of them.

In this sense, we might be at the boundary between pre-paradigmatic phase and a period of normal science in the categorical study of systems.
This is conditional on the ability of CST to establish itself as the dominant paradigm. In fact, the subject is still in its infancy.

At the moment, most of CST lives in Myers' own book \cite{myers_categorical_2022}, itself a longer version of the shorter preprint \cite{myers_double_2020} (where the notion of \emph{doctrine} wasn't yet developed).
Moreover, Myers has given a few talks about the topic in the past years:
\begin{enumerate}
	\item \fullcite{myers2020talk1}
	\item \fullcite{myers2020talk2}
	\item \fullcite{myers2020act_talk}
	\item \fullcite{myers2021talk}
\end{enumerate}
I also gave a talk about CST and its extension to cybernetic systems:
\begin{enumerate}[resume]
	\item \fullcite{capucci2022talk}
\end{enumerate}

\subsection{A quick tour of CST}
Categorical systems theory is a conceptually simple, if mathematically sophisticated, framework.
In a nutshell, it studies processes connecting systems, and the ways these behave. Processes are organized in (monoidal double) categories, which themselves index categories of systems, whose maps in (the behavioural theory of) sets are behaviours.

If one is not at ease with double categories, at a first approximation one can drop the horizontal direction and think of these as monoidal categories of processes. They index sets of systems which can be reindexed by processes. One can study behaviour by specifying the set of ways interfaces can be observed, the relations processes induce between observations on their interfaces, and the states systems can be in and the observables these expose.

However, none of the two dimensions in CST is ancillary to the other.
The horizontal direction is often overlooked in pre-CST work, but it's extremely natural to consider: from a categorical standpoint, we study things (here, systems and processes) by looking at the way they map into each other.

Thus the first step in CST is to understand \textbf{processes organize in monoidal double categories}:
\begin{equation}
	\Processes := \left\{
		% https://q.uiver.app/?q=WzAsNCxbMCwwLCJcXGJ1bGxldCJdLFszLDAsIlxcYnVsbGV0Il0sWzAsMiwiXFxidWxsZXQiXSxbMywyLCJcXGJ1bGxldCJdLFswLDIsIlxcdGV4dHtwcm9jZXNzfSIsMV0sWzEsMywiXFx0ZXh0e3Byb2Nlc3N9IiwxXSxbMCwxLCJcXHRleHR7bWFwIG9mIGludGVyZmFjZXN9IiwxXSxbMiwzLCJcXHRleHR7bWFwIG9mIGludGVyZmFjZXN9IiwxXSxbNCw1LCJcXHRleHR7bWFwIG9mIHByb2Nlc3Nlc30iLDAseyJzaG9ydGVuIjp7InNvdXJjZSI6MjAsInRhcmdldCI6MjB9fV1d
		{\scriptstyle
		\begin{tikzcd}[ampersand replacement=\&, column sep=normal]
			\cdot \&\&\& \cdot \\
			\\
			\cdot \&\&\& \cdot
			\arrow[""{name=0, anchor=center, inner sep=0}, "{\text{process}}"{description}, from=1-1, to=3-1]
			\arrow[""{name=1, anchor=center, inner sep=0}, "{\text{process}}"{description}, from=1-4, to=3-4]
			\arrow["{\text{map of interfaces}}"{description}, from=1-1, to=1-4]
			\arrow["{\text{map of interfaces}}"{description}, from=3-1, to=3-4]
			\arrow["{\text{map of processes}}", shift right=1, shorten <=19pt, shorten >=19pt, Rightarrow, from=0, to=1]
		\end{tikzcd}}
	\right\}
\end{equation}
Such processes are actually `composition patterns' that can be used to weave systems together, i.e.~the ways parts can come together to form wholes. These can be wiring diagrams, or bubble diagrams, or circuit diagrams, etc. Both ways of thinking about them can be useful.

Mathematically speaking, \textbf{processes index systems}, giving rise to doubly indexed categories called \textbf{systems theories}:
\begin{equation}
	\Sys : \Processes^\top \unilaxto \dblCat
\end{equation}
Thus, and this is a fundamental idea in CST, systems and processes are formally distinguished, even though they might end up being quite similar. In fact, in many instances, systems are special instances of processes which are considered stateful.
The categories of systems over a given interface are categories of structure-preserving morphisms of systems, which we call simulations here. These can be more or less rigid depending on the user's taste.

Finally, systems are as interesting as the things they do.
The observations we can make of a system are its behaviour. Ways to observe systems in a given theory are \textbf{theories of behaviour}, which are maps into the `behavioural theory':
\begin{equation}
	% https://q.uiver.app/?q=WzAsNSxbMCwwLCJcXFByb2Nlc3Nlc15cXHRvcCJdLFswLDIsIlxcU3BhbihcXGNhdCBDKV5cXHRvcCJdLFsyLDEsIlxcZGJsQ2F0Il0sWzEsMF0sWzEsMl0sWzAsMSwiQl5cXHRvcCIsMl0sWzAsMiwiXFxTeXMiLDAseyJjdXJ2ZSI6LTF9XSxbMSwyLCJcXGNhdCBDLy0iLDIseyJjdXJ2ZSI6MX1dLFszLDQsIkJeXFxmbGF0IiwwLHsib2Zmc2V0Ijo1LCJzaG9ydGVuIjp7InNvdXJjZSI6MzAsInRhcmdldCI6MzB9LCJsZXZlbCI6Mn1dXQ==
	\begin{tikzcd}[ampersand replacement=\&, sep=small]
		{\Processes^\top} \& {} \\
		\&\& \dblCat \\
		{\dblSet^\top} \& {}
		\arrow["{B^\top}"', from=1-1, to=3-1]
		\arrow["\Sys", curve={height=-6pt}, from=1-1, to=2-3]
		\arrow["{\Set/-}"', curve={height=6pt}, from=3-1, to=2-3]
		\arrow["{B^\flat}"', shift right=4, shorten <=10pt, shorten >=15pt, Rightarrow, from=1-2, to=3-2]
	\end{tikzcd}
\end{equation}
Here $\dblSet$ is the double category of spans in $\Set$ and $\dblCat$ is the double category of functors and profunctors.

\subsection{Prerequisites}
CST is deeply rooted in double category theory. This is dictated by the structure of processes: they compose like morphisms but are also the subject of morphisms.

For many notions, we will reference \cite{grandis_higher_2019}.
For a slow paced, well-motivated introduction of the minimum double category theory used in CST, we invite the reader to read along the main reference \cite{myers_categorical_2022}.

\subsection{Acknowledgments}
Some of the wisdom collected in these notes is not mine, but comes from fruitful conversations. Chiefly, David Jaz Myers. Among them: Ezra Schoen (helped greatly with~\cref{ex:coalgebras}), Nima Motamed, and Nathaniel Virgo.
